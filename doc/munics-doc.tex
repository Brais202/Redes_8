\documentclass{ltxdockit}
\usepackage{btxdockit}
\usepackage[utf8]{inputenc}
\usepackage[spanish]{babel}
\usepackage[strict]{csquotes}
\usepackage{tabularx}
\usepackage{longtable}
\usepackage{booktabs}
\usepackage{shortvrb}
\usepackage{pifont}
\usepackage{graphicx}
\usepackage{hyperref}
\usepackage{listings}
\usepackage{xcolor}
\usepackage{float}
\newcommand{\pathServ}{images/servicios/}
\newcommand{\pathAcls}{images/acls/}
\newcommand{\pathNat}{images/nat/}
\newcommand{\pathZbfw}{images/zbfw/}
\newcommand{\pathTests}{images/pruebas/}
\newcommand{\pathRoot}{images/}



% Definición de colores para código
\definecolor{codegreen}{rgb}{0,0.6,0}
\definecolor{codegray}{rgb}{0.5,0.5,0.5}
\definecolor{codepurple}{rgb}{0.58,0,0.82}
\definecolor{backcolour}{rgb}{0.95,0.95,0.92}

\lstdefinestyle{bashstyle}{
	backgroundcolor=\color{backcolour},   
	commentstyle=\color{codegreen},
	keywordstyle=\color{magenta},
	numberstyle=\tiny\color{codegray},
	stringstyle=\color{codepurple},
	basicstyle=\ttfamily\footnotesize,
	breakatwhitespace=false,         
	breaklines=true,                 
	captionpos=b,                    
	keepspaces=true,                 
	numbers=left,                    
	numbersep=5pt,                  
	showspaces=false,                
	showstringspaces=false,
	showtabs=false,                  
	tabsize=2,
	frame=single,
	language=bash
}

\lstset{style=bashstyle}

\newcommand*{\munics}{\emph{MUniCS Master's Thesis}\xspace}

\titlepage{%
	title={SEGURIDAD DE REDES},
	subtitle={Laboratorio 8 - Seguridad Perimetral},
	url={},
	author={Estela Pillo González \\ Orlando J. Garcés Casal \\ Simón Noya Dominguez \\ Alvaro Cainzos Urtiaga \\ Brais Gómez Espiñeira},
	email={estela.pgonzalez@udc.es \\ o.garces@udc.es \\ simon.noyad@udc.es \\ alvaro.cainzos@udc.es \\ brais.gomez2@udc.es},
	revision={1.0},
	date={\today}}

\hypersetup{%
	pdftitle={Documentación de Laboratorio 8 - Seguridad Perimetral},
	pdfsubject={Laboratorio 8 - Seguridad Perimetral},
	pdfauthor={Nombre del Estudiante},
	pdfkeywords={redes, seguridad, firewall, ACL, NAT, perimetral}}

\begin{document}
	
	\printtitlepage
	\tableofcontents
	
	\section{Introducción}
	\label{sec:intro}
	
	A continuación se presenta la memoria de este laboratorio, donde se explica cómo se desarrolló toda la práctica paso a paso. 
	En este documento se describe la configuración realizada en cada parte del entorno, incluyendo la puesta en marcha de los
	servidores, la aplicación de filtros y reglas de seguridad, la configuración del firewall con distintas técnicas, 
	la implementación de NAT y port forwarding, así como las pruebas finales de conectividad para verificar que todo 
	funciona correctamente.
	
	\section{Laboratorio 8: Seguridad Perimetral}
	\label{sec:lab8}
	
	\subsection{Parte 1: Despliegue de Servicios}
	\label{sec:lab8:parte1}
	
	En esta primera parte del laboratorio se preparó la infraestructura básica necesaria para que las distintas VLAN del entorno pudieran acceder a servicios internos comunes. 
	Para ello, en la VLAN de servicios se desplegaron tres componentes principales: un servidor web accesible tanto por HTTP como por HTTPS, y un servidor DNS encargado de resolver los nombres del dominio interno. 

	\subsubsection{Configuración del servidor HTTP y HTTPS}
	\label{sec:lab8:servidores:http}
	
	Se decidió configurar el servidor web sobre \texttt{nginx}.\\
	Para habilitar HTTPS fue necesaria la creación de un certificado SSL autofirmado por nginx. Este procedimiento se muestra en la Figura \ref{fig:nginx1}:
	
	\begin{figure}[H]
		\centering
		\includegraphics[width=\textwidth]{\pathServ nginx_config_1.png}
		\caption{Generación y firma del certificado SSL.}
		\label{fig:nginx1}
	\end{figure}
	
	A continuación creamos y configuramos las carpetas de la web. En la Figura \ref{fig:nginx2} se muestran los distintos comandos utilizados para la creación y configuración de la carpeta de la web. En conjunto se realiza lo siguiente:
	\begin{itemize}
		\item Creación de la carpeta del sitio web.
		\item Se proporcionan permisos para que nginx sea capaz de acceder.
		\item Creación del archivo de configuración de la web.
		\item Link de los archivos a sites-enabled.
		\item Comprobación de la configuración de nginx.
	\end{itemize}
	
	\begin{figure}[H]
		\centering
		\includegraphics[width=\textwidth]{\pathServ nginx_config_2.png}
		\caption{Creación de archivos necesarios para el servidor.}
		\label{fig:nginx2}
	\end{figure}
	
	Es necesario configurar el servidor web para manejar el tráfico HTTP y HTTPS. En la Figura \ref{fig:nginx_http} se muestra la configuración para manejar el tráfico HTTP.
	
	\begin{figure}[H]
		\centering
		\includegraphics[width=\textwidth]{\pathServ nginx_config_http.png}
		\caption{Configuración nginx HTTP.}
		\label{fig:nginx_http}
	\end{figure}
	

	El servicio HTTPS se configuró tal como muestra la Figura \ref{fig:nginx_https}. 
	
	\begin{figure}[H]
		\centering
		\includegraphics[width=\textwidth]{\pathServ nginx_config_https.png}
		\caption{Configuración nginx HTTPS.}
		\label{fig:nginx_https}
	\end{figure}
	
	El contenido publicado consistió en una página HTML sencilla, mostrada en la Figura \ref{fig:nginx_html}:
	
	\begin{figure}[H]
		\centering
		\includegraphics[width=\textwidth]{\pathServ nginx_config_indexhtml.png}
		\caption{Contenido del archivo \texttt{index.html}}
		\label{fig:nginx_html}
	\end{figure}
	
	Finalmente, tal como se muestra en la Figura \ref{fig:nginx_acceso}, se comprueba que los clientes de la red podían acceder correctamente al servidor web.
	
	\begin{figure}[H]
		\centering
		\includegraphics[width=\textwidth]{\pathServ nginx_check_works.png}
		\caption{Acceso al servidor web.}
		\label{fig:nginx_acceso}
	\end{figure}
	
	\subsubsection{Configuración del Servidor DNS}
	\label{sec:lab8:servidores:dns}
	
	Para proporcionar resolución de nombres dentro de la red, se configuró un servidor DNS mediante \texttt{bind9}.
	
	Para el funcionamiento del servidor DNS se requirieron los siguientes pasos:
	\begin{itemize}
		\item Configurar las opciones principales del servidor DNS (\ref{fig:dns1}).
		\item Definir la zona DNS que maneja el servidor (\ref{fig:dns})).
		\item Creación de archivo de zona directa (\ref{fig:dns3})
		\end{itemize}
	
	\begin{figure}[H]
		\centering
		\includegraphics[width=\textwidth]{\pathServ DNS_1.png}
		\caption{Configuración inicial del servidor DNS.}
		\label{fig:dns1}
		
	\end{figure}
	
	\begin{figure}[H]
		\centering
		\includegraphics[width=\textwidth]{\pathServ DNS_2.png}
		\caption{Definición de la zona DNS interna.}
		\label{fig:dns2}
	\end{figure}
	
	\begin{figure}[H]
		\centering
		\includegraphics[width=\textwidth]{\pathServ DNS_3.png}
		\caption{Creación y definición de archivos de zona.}
		\label{fig:dns3}
	\end{figure}
	
	En la Figura \ref{fig:dns4} se comprueba todo lo implementado y el estado actual del servicio.
	\begin{figure}[H]
		\centering
		\includegraphics[width=\textwidth]{\pathServ DNS_4.png}
		\caption{Comprobación del funcionamiento del servicio.}
		\label{fig:dns4}
	\end{figure}
	
	En la Figura \ref{fig:dns5} de muestra una prueba de la resolución DNS y acceso mediante HTTP.
	\begin{figure}[H]
		\centering
		\includegraphics[width=\textwidth]{\pathServ DNS_5.png}
		\caption{Comprobación del funcionamiento del servicio, parte 2.}
		\label{fig:dns5}
	\end{figure}
	
	Por último, se comprueba el funcionamiento del servicio HTTPS en la Figura \ref{fig:dns6}.
	\begin{figure}[H]
		\centering
		\includegraphics[width=\textwidth]{\pathServ DNS_6.png}
		\caption{Comprobación del funcionamiento del servicio, parte 3.}
		\label{fig:dns6}
	\end{figure}
	
	\subsection{Parte 2: Configuración de Filtros en los Firewall}
	\label{sec:lab8:parte2}
	
	\subsubsection{Filtrado Estático de Paquetes}
	\label{sec:lab8:filtrado-estatico}
	
	En el router CPE se implementó una ACL estándar (ACL 10) en la interfaz externa (GigabitEthernet0/1) para filtrar paquetes con direcciones IP de origen no válidas o peligrosas. Esta ACL se aplica en dirección de entrada para proteger la red interna de ataques comunes como IP spoofing.
	
	\begin{itemize}
		\item \textbf{Direcciones privadas:} Se bloquean todos los rangos RFC 1918:
		\begin{itemize}
			\item 10.0.0.0/8
			\item 172.16.0.0/12
			\item 192.168.0.0/16
		\end{itemize}
		
		\item \textbf{Direcciones de enlace local:} Se bloquea el rango 169.254.0.0/16
		
		\item \textbf{Direcciones multicast:} Se bloquea el rango 224.0.0.0/4
		
		\item \textbf{Direcciones de broadcast:} Se bloquea la dirección 255.255.255.255
		
		\item \textbf{Loopback:} Se bloquea el rango 127.0.0.0/8
		
		\item \textbf{Otras direcciones peligrosas:}
		\begin{itemize}
			\item Dirección cero (0.0.0.0/8) - usada en ataques de redirección
			\item Rango 100.64.0.0/10 - direcciones de CGN (Carrier-Grade NAT)
			\item 192.88.99.0/24 - anycast IPv6 to IPv4
			\item 198.18.0.0/15 - pruebas de benchmark
			\item 198.51.100.0/24 - documentación (TEST-NET-2)
			\item 203.0.113.0/24 - documentación (TEST-NET-3)
			\item Rango 240.0.0.0/4 - direcciones reservadas/futuras
		\end{itemize}
	\end{itemize}
	
	La ACL 10 está diseñada siguiendo las mejores prácticas de seguridad (BCP 38/RFC 2827) para prevenir IP spoofing. Se aplica en la interfaz externa con la política de "deny-by-default", permitiendo solo tráfico legítimo al final de la lista.
	
	\begin{lstlisting}[language=bash, caption=ACL 10 - Filtrado estático en CPE, label=lst:acl10-cpe]
		! ACL aplicada en GigabitEthernet0/1 (interfaz externa)
		interface GigabitEthernet0/1
		ip access-group 10 in  ! Filtrado en dirección de entrada
		
		! ACL estándar 10 - Filtrado anti-spoofing
		access-list 10 deny   255.255.255.255      ! Dirección de broadcast
		access-list 10 deny   0.0.0.0 0.255.255.255 ! Dirección cero
		access-list 10 deny   10.0.0.0 0.255.255.255 ! Direcciones privadas clase A
		access-list 10 deny   100.64.0.0 0.63.255.255 ! CGN (RFC 6598)
		access-list 10 deny   127.0.0.0 0.255.255.255 ! Loopback
		access-list 10 deny   169.254.0.0 0.0.255.255 ! Link-local (APIPA)
		access-list 10 deny   172.16.0.0 0.15.255.255 ! Direcciones privadas clase B
		access-list 10 deny   192.88.99.0 0.0.0.255   ! Anycast 6to4
		access-list 10 deny   192.168.0.0 0.0.255.255 ! Direcciones privadas clase C
		access-list 10 deny   198.18.0.0 0.1.255.255  ! Benchmark testing
		access-list 10 deny   198.51.100.0 0.0.0.255  ! TEST-NET-2
		access-list 10 deny   203.0.113.0 0.0.0.255   ! TEST-NET-3
		access-list 10 deny   224.0.0.0 15.255.255.255 ! Multicast (224.0.0.0/4)
		access-list 10 deny   240.0.0.0 15.255.255.255 ! Reservado/Futuro (240.0.0.0/4)
		access-list 10 permit any                     ! Permitir tráfico legítimo
	\end{lstlisting}
	
	\subsubsection{Configuración de CBAC en FW}
	\label{sec:lab8:cbac}
	
	La implementación de la política de seguridad definida en la  tabla del enuncaido se realizó mediante Context-Based Access Control (CBAC) en el firewall perimetral.
	
	
	
	\paragraph{Configuración de CBAC:}
	Se creó una inspección CBAC llamada "CBAC-FW" que examina los protocolos TCP, UDP, ICMP, HTTP, HTTPS y SSH para mantener el estado de las conexiones.
	
	\begin{lstlisting}[language=bash, caption=Configuración CBAC en el firewall, label=lst:cbac-config]
		! Definición de la inspección CBAC
		ip inspect name CBAC-FW tcp    ! Inspección de conexiones TCP
		ip inspect name CBAC-FW udp    ! Inspección de sesiones UDP
		ip inspect name CBAC-FW icmp   ! Inspección de paquetes ICMP
		ip inspect name CBAC-FW http   ! Inspección específica para HTTP
		ip inspect name CBAC-FW https  ! Inspección específica para HTTPS
		ip inspect name CBAC-FW ssh    ! Inspección específica para SSH
		
		! Aplicación de CBAC en las interfaces
		interface GigabitEthernet0/0.2    ! Hacia red interna
		ip inspect CBAC-FW in
		
		interface GigabitEthernet0/0.746  ! Hacia DMZ (servidores)
		ip inspect CBAC-FW in
		
		interface GigabitEthernet0/1      ! Hacia Internet
		ip inspect CBAC-FW in
	\end{lstlisting}
	
	\paragraph{ACLs Extendidas para Política de Seguridad:}
	Se implementaron tres ACLs extendidas principales:
	
	\begin{enumerate}
		\item \textbf{FW\_INTERNET\_IN:} Controla el tráfico entrante desde Internet
		\item \textbf{lista\_CBAC:} Controla el tráfico entre redes internas y hacia Internet
		\item \textbf{lista\_dmz:} Controla el tráfico desde la DMZ hacia Internet
	\end{enumerate}
	
	\begin{lstlisting}[language=bash, caption=ACL extendida lista\_CBAC (política principal), label=lst:lista-cbac]
		ip access-list extended lista_CBAC
		! VLANs -> Servicios (DNS, HTTP, HTTPS, ICMP)
		permit udp 10.4.16.0 0.0.0.255 10.4.246.0 0.0.0.255 eq domain  ! VLAN16 a DNS
		permit tcp 10.4.16.0 0.0.0.255 10.4.246.0 0.0.0.255 eq www      ! VLAN16 a HTTP
		permit tcp 10.4.16.0 0.0.0.255 10.4.246.0 0.0.0.255 eq 443      ! VLAN16 a HTTPS
		permit icmp 10.4.16.0 0.0.0.255 10.4.246.0 0.0.0.255           ! VLAN16 ICMP
		permit tcp 10.4.16.0 0.0.0.255 10.4.246.0 0.0.0.255 eq domain  ! VLAN16 a DNS/TCP
		
		! VLANs -> Internet (HTTP, HTTPS, DNS, ICMP)
		permit tcp 10.4.16.0 0.0.0.255 any eq www    ! VLAN16 HTTP saliente
		permit tcp 10.4.16.0 0.0.0.255 any eq 443    ! VLAN16 HTTPS saliente
		permit udp 10.4.16.0 0.0.0.255 any eq domain ! VLAN16 DNS saliente
		permit icmp 10.4.16.0 0.0.0.255 any          ! VLAN16 ICMP saliente
		
		! Se repiten reglas similares para VLAN17 y VLAN18...
		
		! ADMIN -> SSH anywhere
		permit tcp 10.4.245.0 0.0.0.255 any eq 22    ! SSH desde administración
		
		! Permitir OSPF para enrutamiento
		permit ospf host 10.4.0.1 host 224.0.0.5     ! OSPF multicast
		permit ospf host 10.4.0.1 host 10.4.0.2      ! OSPF unicast
		permit ospf host 10.4.0.2 host 224.0.0.5
		permit ospf host 10.4.0.2 host 10.4.0.1
		
		! Reglas de retorno para sesiones establecidas (CBAC)
		permit tcp any any established                ! Conexiones TCP establecidas
		permit udp any any eq domain                  ! Respuestas DNS
		permit icmp any any echo-reply                ! Respuestas ping
		permit icmp any any time-exceeded             ! TTL exceeded
		permit icmp any any unreachable               ! Destino inalcanzable
		permit icmp any any echo                      ! Pings salientes
		
		! Denegar todo lo demás con logging
		deny ip any any log
	\end{lstlisting}
	
	\paragraph{ACL para Tráfico desde Internet:}
	\begin{lstlisting}[language=bash, caption=ACL FW\_INTERNET\_IN, label=lst:fw-internet-in]
		ip access-list extended FW_INTERNET_IN
		! Internet a Servicios (HTTPS específico + ICMP)
		permit icmp any 10.4.246.0 0.0.0.255        ! Ping a servidores
		permit tcp any host 10.4.246.38 eq 443      ! HTTPS a servidor específico
		
		! Permitir OSPF para enrutamiento
		permit ospf host 10.4.0.5 host 224.0.0.5    ! OSPF con CPE
		permit ospf host 10.4.0.5 host 10.4.0.6
		permit ospf host 10.4.0.6 host 224.0.0.5
		permit ospf host 10.4.0.6 host 10.4.0.5
		
		! Denegar todo lo demás con logging
		deny ip any any log
	\end{lstlisting}
	
	\paragraph{ACL para la DMZ:}
	\begin{lstlisting}[language=bash, caption=ACL lista\_dmz, label=lst:lista-dmz]
		ip access-list extended lista_dmz
		! Servicios hacia Internet (respuestas)
		permit tcp 10.4.246.0 0.0.0.255 eq www any   ! HTTP desde servidores
		permit tcp 10.4.246.0 0.0.0.255 eq 443 any   ! HTTPS desde servidores
		permit udp 10.4.246.0 0.0.0.255 eq domain any ! DNS desde servidores
		permit icmp 10.4.246.0 0.0.0.255 any         ! ICMP desde servidores
		
		! Respuestas a pings entrantes
		permit icmp any 10.4.246.0 0.0.0.255 echo-reply
		
		! Conexiones establecidas
		permit tcp any any established
		
		! Denegar todo lo demás
		deny ip any any log
	\end{lstlisting}
	
	\paragraph{Aplicación de las ACLs:}
	Las ACLs se aplicaron en las interfaces correspondientes para implementar la política de seguridad:
	
	\begin{lstlisting}[language=bash, caption=Aplicación de ACLs en interfaces, label=lst:aplicacion-acls]
		interface GigabitEthernet0/0.2    ! Hacia red interna (DL-SW1)
		ip access-group lista_CBAC in    ! Filtrado de entrada
		
		interface GigabitEthernet0/0.746  ! Hacia DMZ (servidores)
		ip access-group lista_dmz in     ! Filtrado de entrada
		
		interface GigabitEthernet0/1      ! Hacia Internet (CPE)
		ip access-group FW_INTERNET_IN in ! Filtrado de entrada desde Internet
	\end{lstlisting}
	
	Esta configuración implementa completamente la política de seguridad definida en la Tabla 1, permitiendo el tráfico autorizado entre zonas mientras bloquea todo el tráfico no permitido explícitamente. El uso de CBAC garantiza que las conexiones establecidas legítimamente puedan recibir tráfico de retorno sin necesidad de reglas explícitas para cada flujo bidireccional.
	
	\subsubsection{Configuración de Zone-Based Firewall en FW}
	\label{sec:lab8:zbfw}
	
	Implementación alternativa de la política de seguridad mediante ZBFW:
	\begin{itemize}
		\item Definición de zonas (inside, outside, DMZ, etc.).
		\item Políticas entre zonas.
		\item Comparación CBAC vs ZBFW (ventajas/desventajas en esta topología).
	\end{itemize}
	
	\paragraph{ACLs para Políticas de Zona:}
	Se implementaron cinco ACLs extendidas para definir el tráfico permitido entre zonas:
	
	\subparagraph{ACL para tráfico desde INSIDE hacia OUTSIDE:}
	Permite a las VLANs de usuarios acceder a servicios de Internet.
	
	\begin{lstlisting}[language=bash, caption=ACL-INSIDE-TO-OUTSIDE, label=lst:acl-inside-outside]
		ip access-list extended ACL-INSIDE-TO-OUTSIDE
		remark VLAN16 -> Internet
		permit tcp 10.4.16.0 0.0.0.255 any eq www
		permit tcp 10.4.16.0 0.0.0.255 any eq 443
		permit udp 10.4.16.0 0.0.0.255 any eq domain
		permit tcp 10.4.16.0 0.0.0.255 any eq domain
		permit icmp 10.4.16.0 0.0.0.255 any
		remark VLAN17 -> Internet
		permit tcp 10.4.17.0 0.0.0.255 any eq www
		permit tcp 10.4.17.0 0.0.0.255 any eq 443
		permit udp 10.4.17.0 0.0.0.255 any eq domain
		permit tcp 10.4.17.0 0.0.0.255 any eq domain
		permit icmp 10.4.17.0 0.0.0.255 any
		remark VLAN18 -> Internet
		permit tcp 10.4.18.0 0.0.0.255 any eq www
		permit tcp 10.4.18.0 0.0.0.255 any eq 443
		permit udp 10.4.18.0 0.0.0.255 any eq domain
		permit tcp 10.4.18.0 0.0.0.255 any eq domain
		permit icmp 10.4.18.0 0.0.0.255 any
	\end{lstlisting}
	
	\subparagraph{ACL para tráfico desde INSIDE hacia DMZ:}
	Permite a las VLANs de usuarios acceder a los servidores internos.
	
	\begin{lstlisting}[language=bash, caption=ACL-INSIDE-TO-DMZ, label=lst:acl-inside-dmz]
		ip access-list extended ACL-INSIDE-TO-DMZ
		remark VLAN16 -> DMZ (Servicios)
		permit tcp 10.4.16.0 0.0.0.255 10.4.246.0 0.0.0.255 eq www
		permit tcp 10.4.16.0 0.0.0.255 10.4.246.0 0.0.0.255 eq 443
		permit udp 10.4.16.0 0.0.0.255 10.4.246.0 0.0.0.255 eq domain
		permit tcp 10.4.16.0 0.0.0.255 10.4.246.0 0.0.0.255 eq domain
		permit icmp 10.4.16.0 0.0.0.255 10.4.246.0 0.0.0.255
		remark VLAN17 -> DMZ (Servicios)
		permit tcp 10.4.17.0 0.0.0.255 10.4.246.0 0.0.0.255 eq www
		permit tcp 10.4.17.0 0.0.0.255 10.4.246.0 0.0.0.255 eq 443
		permit udp 10.4.17.0 0.0.0.255 10.4.246.0 0.0.0.255 eq domain
		permit tcp 10.4.17.0 0.0.0.255 10.4.246.0 0.0.0.255 eq domain
		permit icmp 10.4.17.0 0.0.0.255 10.4.246.0 0.0.0.255
		remark VLAN18 -> DMZ (Servicios)
		permit tcp 10.4.18.0 0.0.0.255 10.4.246.0 0.0.0.255 eq www
		permit tcp 10.4.18.0 0.0.0.255 10.4.246.0 0.0.0.255 eq 443
		permit udp 10.4.18.0 0.0.0.255 10.4.246.0 0.0.0.255 eq domain
		permit tcp 10.4.18.0 0.0.0.255 10.4.246.0 0.0.0.255 eq domain
		permit icmp 10.4.18.0 0.0.0.255 10.4.246.0 0.0.0.255
		remark VLAN16 -> DMZ (HTTP/HTTPS/DNS/ICMP)
		remark VLAN17 -> DMZ
		remark VLAN18 -> DMZ
	\end{lstlisting}
	
	\subparagraph{ACL para tráfico desde OUTSIDE hacia DMZ:}
	Permite acceso controlado desde Internet a servicios específicos en la DMZ.
	
	\begin{lstlisting}[language=bash, caption=ACL-OUTSIDE-TO-DMZ, label=lst:acl-outside-dmz]
		ip access-list extended ACL-OUTSIDE-TO-DMZ
		remark Internet -> Servidor HTTPS en DMZ + Ping
		permit icmp any host 10.4.246.38
		permit tcp any host 10.4.246.38 eq 443
	\end{lstlisting}
	
	\subparagraph{ACL para tráfico desde DMZ hacia OUTSIDE:}
	Permite a los servidores responder a solicitudes y acceder a Internet para actualizaciones.
	
	\begin{lstlisting}[language=bash, caption=ACL-DMZ-TO-OUTSIDE, label=lst:acl-dmz-outside]
		ip access-list extended ACL-DMZ-TO-OUTSIDE
		remark DMZ -> Internet (HTTP/HTTPS/DNS/ICMP)
		permit tcp 10.4.246.0 0.0.0.255 any eq www
		permit tcp 10.4.246.0 0.0.0.255 any eq 443
		permit udp 10.4.246.0 0.0.0.255 any eq domain
		permit icmp 10.4.246.0 0.0.0.255 any
	\end{lstlisting}
	
	\subparagraph{ACL para protocolos de enrutamiento OSPF:}
	Permite el tráfico OSPF necesario para mantener la convergencia de la red.
	
	\begin{lstlisting}[language=bash, caption=ACL-OSPF, label=lst:acl-ospf]
		ip access-list extended ACL-OSPF
		remark OSPF entre todos los routers
		permit ospf host 10.4.0.1 host 224.0.0.5
		permit ospf host 10.4.0.1 host 10.4.0.2
		permit ospf host 10.4.0.2 host 224.0.0.5
		permit ospf host 10.4.0.2 host 10.4.0.1
		permit ospf host 10.4.0.5 host 224.0.0.5
		permit ospf host 10.4.0.5 host 10.4.0.6
		permit ospf host 10.4.0.6 host 224.0.0.5
		permit ospf host 10.4.0.6 host 10.4.0.5
	\end{lstlisting}
	
	\begin{figure}[H]
		\centering
		\includegraphics[width=\textwidth]{\pathZbfw class maps.png}
		\caption{Class-maps para inspección de tráfico en el Zone-Based Firewall.}
		\label{fig:zbfw_class_maps}
	\end{figure}
	
	\begin{figure}[H]
		\centering
		\includegraphics[width=\textwidth]{\pathZbfw policy map inside to outside .png}
		\caption{Policy-map para tráfico \emph{inside} $\rightarrow$ \emph{outside}.}
		\label{fig:zbfw_policy_inside_outside}
	\end{figure}
	
	\begin{figure}[H]
		\centering
		\includegraphics[width=\textwidth]{\pathZbfw policy map inside to dmz.png}
		\caption{Policy-map para tráfico \emph{inside} $\rightarrow$ DMZ.}
		\label{fig:zbfw_policy_inside_dmz}
	\end{figure}
	
	\begin{figure}[H]
		\centering
		\includegraphics[width=\textwidth]{\pathZbfw policy map dmz to outside.png}
		\caption{Policy-map para tráfico DMZ $\rightarrow$ \emph{outside}.}
		\label{fig:zbfw_policy_dmz_outside}
	\end{figure}
	
	\begin{figure}[H]
		\centering
		\includegraphics[width=\textwidth]{\pathZbfw policy map outside to dmz.png}
		\caption{Policy-map para tráfico \emph{outside} $\rightarrow$ DMZ.}
		\label{fig:zbfw_policy_outside_dmz}
	\end{figure}
	
	\begin{figure}[H]
		\centering
		\includegraphics[width=\textwidth]{\pathZbfw zone pair in-out.png}
		\caption{Zone-pair entre las zonas \emph{inside} y \emph{outside}.}
		\label{fig:zbfw_zonepair_in_out}
	\end{figure}
	
	\begin{figure}[H]
		\centering
		\includegraphics[width=\textwidth]{\pathZbfw zone pair in-dmz.png}
		\caption{Zone-pair entre \emph{inside} y DMZ.}
		\label{fig:zbfw_zonepair_in_dmz}
	\end{figure}
	
	\begin{figure}[H]
		\centering
		\includegraphics[width=\textwidth]{\pathZbfw zone pair dmz-out.png}
		\caption{Zone-pair entre DMZ y \emph{outside}.}
		\label{fig:zbfw_zonepair_dmz_out}
	\end{figure}
	
	\begin{figure}[H]
		\centering
		\includegraphics[width=\textwidth]{\pathZbfw zone pair out-dmz.png}
		\caption{Zone-pair entre \emph{outside} y DMZ.}
		\label{fig:zbfw_zonepair_out_dmz}
	\end{figure}
	
	\begin{figure}[H]
		\centering
		\includegraphics[width=\textwidth]{\pathZbfw zone securities.png}
		\caption{Resumen de las zonas de seguridad configuradas en el firewall.}
		\label{fig:zbfw_zone_securities}
	\end{figure}
	
	\begin{figure}[H]
		\centering
		\includegraphics[width=\textwidth]{\pathZbfw SSH_ZBFW.png}
		\caption{Comprobación de acceso SSH protegido por el Zone-Based Firewall.}
		\label{fig:zbfw_ssh}
	\end{figure}
	
	\begin{figure}[H]
		\centering
		\includegraphics[width=\textwidth]{\pathZbfw asignacion zonas a interfaces.png}
		\caption{Asignación de interfaces físicas a zonas de seguridad.}
		\label{fig:zbfw_zonas_interfaces}
	\end{figure}
	
	\newpage
	
	\subsection*{Elección entre CBAC o ZBFW}
	
	\begin{table}[h]
		\centering
		\label{tab:comparativa}
		\begin{tabular}{>{\bfseries}l p{5.5cm} p{5.5cm}}
			\toprule
			\textbf{} &    \textbf{CBAC} & \textbf{ZBFW} \\
			\midrule
			Seguridad & Permisivo por defecto & \textbf{Deniega por defecto} (principio deny-all) \\
			\midrule
			Mantenimiento & Reglas bidireccionales complejas & \textbf{Políticas unidireccionales} simples \\
			\midrule
			Escalabilidad & Baja: modificar ACLs por cada cambio & \textbf{Alta:} solo asignar interfaces a zonas \\
			\midrule
			Claridad & Lógica dispersa en múltiples ACLs & \textbf{Visibilidad unificada} en policy-maps \\
			\bottomrule
		\end{tabular}
		\caption{Comparativa clave entre CBAC y ZBFW}
	\end{table}
	
	\subsection*{Motivos principales para elegir ZBFW:}
	
	\begin{itemize}
		\item \textbf{Seguridad robusta:} Bloqueo automático entre zonas no emparejadas
		\item \textbf{Sencillez operativa:} Políticas definidas una vez por par de zonas
		\item \textbf{Escalabilidad:} Añadir VLANs solo requiere asignarlas a zonas existentes
		\item \textbf{Prevención de errores:} Configuraciones olvidadas no comprometen la seguridad
	\end{itemize}
	
	Para esta topología con zonas definidas (INSIDE, DMZ, OUTSIDE), ZBFW proporciona mayor seguridad, menor complejidad y mejor escalabilidad que CBAC.
	
	\subsubsection{Filtrado en DL-SW1}
	\label{sec:lab8:filtrado-switch}
	
	En el switch de distribución DL-SW1 se implementó filtrado de capa 3 mediante ACLs extendidas aplicadas directamente en las interfaces SVI (Switched Virtual Interfaces) de cada VLAN. Este enfoque proporciona control granular del tráfico inter-VLAN a nivel de la capa de distribución, complementando la seguridad perimetral implementada en el firewall.
	
	
	\paragraph{Justificación de la Opción Elegida:}
	Se seleccionaron \textbf{ACLs extendidas aplicadas en interfaces SVI} por las siguientes razones:
	\begin{itemize}
		\item Proporcionan control granular del tráfico inter-VLAN a nivel de capa 3.
		\item Permiten implementar la política de aislamiento entre VLANs definida en la Tabla 1.
		\item Son más adecuadas para controlar el tráfico entre subredes diferentes.
		\item Se integran eficientemente con el enrutamiento inter-VLAN del switch.
		\item Ofrecen mejor rendimiento que VACLs para tráfico entre VLANs.
	\end{itemize}
	
	\paragraph{Configuración Aplicada:}
	Se implementaron tres ACLs extendidas (VLAN16, VLAN17, VLAN18) aplicadas en las respectivas interfaces SVI con política de entrada (in).
	
	\subparagraph{ACL para VLAN 16:}
	Aislamiento de la VLAN de alumnos y control de acceso a servicios.
	
	\begin{lstlisting}[language=bash, caption=ACL VLAN16 - Control de tráfico para VLAN de alumnos, label=lst:acl-vlan16]
		ip access-list extended VLAN16
		permit udp any any eq bootps      ! DHCP server (67)
		permit udp any any eq bootpc      ! DHCP client (68)
		deny   ip 10.4.16.0 0.0.0.255 10.4.17.0 0.0.0.255    ! Bloquear VLAN17
		deny   ip 10.4.16.0 0.0.0.255 10.4.18.0 0.0.0.255    ! Bloquear VLAN18
		deny   ip 10.4.16.0 0.0.0.255 10.4.245.0 0.0.0.255   ! Bloquear VLAN Admin
		deny   ip 10.4.16.0 0.0.0.255 10.4.0.0 0.0.0.3       ! Bloquear enlace FW
		permit tcp 10.4.16.0 0.0.0.255 any eq www            ! HTTP saliente
		permit tcp 10.4.16.0 0.0.0.255 any eq 443            ! HTTPS saliente
		permit udp 10.4.16.0 0.0.0.255 any eq domain         ! DNS UDP saliente
		permit tcp 10.4.16.0 0.0.0.255 any eq domain         ! DNS TCP saliente
		permit icmp 10.4.16.0 0.0.0.255 any                  ! ICMP saliente
	\end{lstlisting}
	
	\subparagraph{ACL para VLAN 17:}
	Aislamiento de la VLAN de PDI y control de acceso a servicios.
	
	\begin{lstlisting}[language=bash, caption=ACL VLAN17 - Control de tráfico para VLAN de PDI, label=lst:acl-vlan17]
		ip access-list extended VLAN17
		permit udp any any eq bootps      ! DHCP server (67)
		permit udp any any eq bootpc      ! DHCP client (68)
		deny   ip 10.4.17.0 0.0.0.255 10.4.16.0 0.0.0.255    ! Bloquear VLAN16
		deny   ip 10.4.17.0 0.0.0.255 10.4.18.0 0.0.0.255    ! Bloquear VLAN18
		deny   ip 10.4.17.0 0.0.0.255 10.4.245.0 0.0.0.255   ! Bloquear VLAN Admin
		deny   ip 10.4.17.0 0.0.0.255 10.4.0.0 0.0.0.3       ! Bloquear enlace FW
		permit tcp 10.4.17.0 0.0.0.255 any eq www            ! HTTP saliente
		permit tcp 10.4.17.0 0.0.0.255 any eq 443            ! HTTPS saliente
		permit udp 10.4.17.0 0.0.0.255 any eq domain         ! DNS UDP saliente
		permit tcp 10.4.17.0 0.0.0.255 any eq domain         ! DNS TCP saliente
		permit icmp 10.4.17.0 0.0.0.255 any                  ! ICMP saliente
	\end{lstlisting}
	
	\subparagraph{ACL para VLAN 18:}
	Aislamiento de la VLAN de PAS y control de acceso a servicios.
	
	\begin{lstlisting}[language=bash, caption=ACL VLAN18 - Control de tráfico para VLAN de PAS, label=lst:acl-vlan18]
		ip access-list extended VLAN18
		permit udp any any eq bootps      ! DHCP server (67)
		permit udp any any eq bootpc      ! DHCP client (68)
		deny   ip 10.4.18.0 0.0.0.255 10.4.16.0 0.0.0.255    ! Bloquear VLAN16
		deny   ip 10.4.18.0 0.0.0.255 10.4.17.0 0.0.0.255    ! Bloquear VLAN17
		deny   ip 10.4.18.0 0.0.0.255 10.4.245.0 0.0.0.255   ! Bloquear VLAN Admin
		deny   ip 10.4.18.0 0.0.0.255 10.4.0.0 0.0.0.3       ! Bloquear enlace FW
		permit tcp 10.4.18.0 0.0.0.255 any eq www            ! HTTP saliente
		permit tcp 10.4.18.0 0.0.0.255 any eq 443            ! HTTPS saliente
		permit udp 10.4.18.0 0.0.0.255 any eq domain         ! DNS UDP saliente
		permit tcp 10.4.18.0 0.0.0.255 any eq domain         ! DNS TCP saliente
		permit icmp 10.4.18.0 0.0.0.255 any                  ! ICMP saliente
	\end{lstlisting}
	
	\paragraph{Aplicación de las ACLs en Interfaces SVI:}
	Las ACLs se aplicaron en las interfaces virtuales de cada VLAN con política de entrada:
	
	\begin{lstlisting}[language=bash, caption=Aplicación de ACLs en interfaces SVI, label=lst:aplicacion-acls-svi]
		interface Vlan16
		ip address 10.4.16.1 255.255.255.0
		ip access-group VLAN16 in      ! Aplicar ACL entrada		ip helper-address 10.4.245.100 ! Reenvío DHCP
		
		interface Vlan17
		ip address 10.4.17.1 255.255.255.0
		ip access-group VLAN17 in      ! Aplicar ACL entrada
		ip helper-address 10.4.245.100 ! Reenvío DHCP
		
		interface Vlan18
		ip address 10.4.18.1 255.255.255.0
		ip access-group VLAN18 in      ! Aplicar ACL entrada
		ip helper-address 10.4.245.100 ! Reenvío DHCP
	\end{lstlisting}
	
	\paragraph{Análisis de la Configuración:}
	\begin{itemize}
		\item \textbf{Aislamiento entre VLANs:} Cada ACL bloquea explícitamente el tráfico hacia las otras VLANs de usuario y hacia la VLAN de administración.
		\item \textbf{Acceso a servicios:} Se permite tráfico HTTP, HTTPS, DNS e ICMP saliente hacia cualquier destino.
		\item \textbf{Excepciones DHCP:} Se permiten puertos 67 y 68 para el funcionamiento de DHCP.
		\item \textbf{Blindaje administrativo:} Se bloquea el acceso directo a la VLAN de administración (745) y al enlace con el firewall.
		\item \textbf{Política implícita:} Por defecto, todo el tráfico no explícitamente permitido es denegado (deny ip any any implícito).
	\end{itemize}
	
	Esta configuración implementa una defensa en profundidad, donde el DL-SW1 actúa como segunda barrera de seguridad que impide la comunicación lateral no autorizada entre segmentos de red, mientras que el firewall controla el tráfico hacia/desde Internet y la DMZ.
	
	\subsection{Parte 3: Configuración de NAT}
	\label{sec:lab8:parte3}
	
	En esta parte del laboratorio se configuró la traducción de direcciones de red (NAT) en el CPE con el objetivo de permitir
	que las distintas VLAN de la organización pudieran acceder a Internet utilizando direcciones IP públicas del rango asignado
	al Pod. Además, se configuró port forwarding para publicar los servicios internos (HTTP y HTTPS) hacia el exterior.
	
	
	\subsubsection{PAT dinámico para tráfico saliente}
	\label{sec:lab8:pat}

	Para implementar este comportamiento, el primer paso fue crear un pool de direcciones para cada VLAN. Estos pools permiten que se asignen direcciones dentro de un rango determinado para cada una de las VLANs. Se muestran en la Figura \ref{fig:config_pat1} junto con las ACL necesarias para identificar el tráfico origen de cada red.
	
	\begin{figure}[H]
		\centering
		\includegraphics[width=\textwidth]{\pathNat config_PAT.png}
		\caption{Creación de pools y ACLs asociadas a cada VLAN.}
		\label{fig:config_pat1}
	\end{figure}
	

	Las ACL generadas permiten clasificar el tráfico según la VLAN de origen, mientras que los pools determinan qué dirección pública se asignará a cada una de ellas. \\
	
	El siguiente paso consistió en asociar cada ACL con su pool correspondiente y activar PAT mediante la opción
	\texttt{overload}. La Figura  \ref{fig:config_pat2} muestra este proceso.
	
	\begin{figure}[H]
		\centering
		\includegraphics[width=\textwidth]{\pathNat config_PAT_2.png}
		\caption{Asociación de ACLs y pools, habilitando NAT overload para cada VLAN.}
		\label{fig:config_pat2}
	\end{figure}
	
	A continuación, en la imagen \ref{fig:nat_interfaces} se configuran las interfaces como inside y outside. La interfaz GigabitEthernet0/0.3 se configura como \texttt{inside}, mientras que la interfaz GigabitEthernet0/1 se configura como \texttt{outside}.
	
	\begin{figure}[H]
		\centering
		\includegraphics[width=\textwidth]{\pathNat asignacion nat a interfaces.png}
		\caption{Asignación de NAT (inside/outside) a las interfaces.}
		\label{fig:nat_interfaces}
	\end{figure}
	
	\subsubsection{Port Forwarding para Servicios Internos}
	\label{sec:lab8:portforwarding}
	Se configuró port forwarding para permitir que clientes externos accedieran
	a los servicios publicados por la organización en la VLAN de servicios. Para ello se ejecutan los comandos de la imagen \ref{fig:port_forwarding}.
	
	\begin{figure}[H]
		\centering
		\includegraphics[width=\textwidth]{\pathNat port forwarding.png}
		\caption{Reglas de port forwarding para publicar servicios internos.}
		\label{fig:port_forwarding}
	\end{figure}
	
	\section{Pruebas y Verificación}
	\label{sec:pruebas}
	
	A continuación de muestran distintas pruebas para comprobar el funcionamiento de la práctica:
	
	\subsection{Pruebas realizadas desde la VLAN 18}
	
	% --- Pruebas de conectividad entre VLANs ---
	\begin{figure}[H]
		\centering
		\includegraphics[width=\textwidth]{\pathTests PRUEBAS_PINGS_VLAN18.png}
		\caption{Pruebas de ping realizadas desde la VLAN 18.}
		\label{fig:pruebas_pings_vlan18}
	\end{figure}

    \subsection{Pruebas realizadas desde la VLAN 17}

    \begin{figure}[H]
    \centering
    \includegraphics[width=0.8\textwidth]{\pathTests VLAN17_ping_vlan16.png}
    \caption{Intento de ping a VLAN 16 bloqueado (aislamiento entre VLANs).}
    \label{fig:vlan17_ping_vlan16_denied}
    \end{figure}
    
    \begin{figure}[H]
        \centering
        \includegraphics[width=0.8\textwidth]{\pathTests VLAN17_ping_vlan18.png}
        \caption{Intento de ping a VLAN 18 bloqueado (aislamiento entre VLANs).}
        \label{fig:vlan17_ping_vlan18_denied}
    \end{figure}
    
    \begin{figure}[H]
        \centering
        \includegraphics[width=0.8\textwidth]{\pathTests VLAN17_ping_administracion.png}
        \caption{Intento de ping a VLAN de administración bloqueado.}
        \label{fig:vlan17_ping_admin_denied}
    \end{figure}

    \begin{figure}[H]
    \centering
    \includegraphics[width=0.8\textwidth]{\pathTests VLAN17_ping_servicios.png}
    \caption{Ping desde VLAN 17 al servidor de servicios.}
    \label{fig:vlan17_ping_servicios}
    \end{figure}

    \begin{figure}[H]
    \centering
    \includegraphics[width=0.8\textwidth]{\pathTests VLAN17_ping_enlace.png}
    \caption{Ping desde VLAN 17 a su gateway (10.4.17.1).}
    \label{fig:vlan17_ping_gateway}
    \end{figure}
    
    \begin{figure}[H]
        \centering
        \includegraphics[width=0.8\textwidth]{\pathTests VLAN17_ping_a_pc_en_vlan17.png}
        \caption{Ping dentro de la misma VLAN 17 (10.4.17.12).}
        \label{fig:vlan17_ping_intra_vlan}
    \end{figure}

    
	\subsection{Pruebas realizadas desde la VLAN 16}
	\begin{figure}[H]
		\centering
		\includegraphics[width=\textwidth]{\pathTests zbfw/vlan16tovlan16ok.png}
		\caption{Tráfico permitido dentro de la misma VLAN (VLAN16).}
		\label{fig:zbfw_vlan16_vlan16_ok}
	\end{figure}
	
	\begin{figure}[H]
		\centering
		\includegraphics[width=\textwidth]{\pathTests zbfw/vlan16_to_core.png}
		\caption{Prueba de conectividad desde VLAN16 hacia el core.}
		\label{fig:zbfw_vlan16_core}
	\end{figure}
	
	\begin{figure}[H]
		\centering
		\includegraphics[width=\textwidth]{\pathTests zbfw/vlan16_to_adm_and_serv.png}
		\caption{Prueba de conectividad desde VLAN16 hacia VLAN de administración y servicios.}
		\label{fig:zbfw_vlan16_adm_serv}
	\end{figure}

	\begin{figure}[H]
		\centering
		\includegraphics[width=\textwidth]{\pathTests zbfw/vlan16tovlan17and18.png}
		\caption{Comprobación de restricciones entre VLAN16, VLAN17 y VLAN18.}
		\label{fig:zbfw_vlan16_vlan17_18}
	\end{figure}
	
	\begin{figure}[H]
		\centering
		\includegraphics[width=\textwidth]{\pathTests zbfw/ping_vlan18_to_gateway.png}
		\caption{Ping desde VLAN18 al gateway, filtrado por el ZBFW.}
		\label{fig:zbfw_vlan18_gateway}
	\end{figure}
	
	
	\subsection{Comprobación de Reglas de Filtrado}
	\label{sec:pruebas:filtrado}
	
	\begin{itemize}
		\item \texttt{show access-lists}
		\item \texttt{show ip nat translations}
		\item \texttt{show zone-pair security}
		\item \texttt{show policy-map type inspect}
	\end{itemize}

    \subsection{Pruebas de NAT y acceso externo}
    \label{sec:pruebas:nat_externo}
    
    Para verificar la correcta configuración del NAT (Network Address Translation) y el port forwarding en el router CPE, se realizaron pruebas desde un equipo conectado a la red del ISP.
    
    \paragraph{Prueba de conectividad a la IP pública:}
    Se verificó la conectividad básica a la dirección IP pública asignada para los servicios (192.0.4.203). Como muestra la Figura \ref{fig:ping_ip_nat}, el ping a esta dirección es exitoso, confirmando que el router CPE responde a las solicitudes ICMP entrantes.
    
    \begin{figure}[H]
        \centering
        \includegraphics[width=0.8\textwidth]{\pathTests Ping_a_ip_nat.png}
        \caption{Ping exitoso a la dirección IP pública 192.0.4.203 desde la red del ISP.}
        \label{fig:ping_ip_nat}
    \end{figure}
    
    \paragraph{Prueba de acceso HTTPS desde Internet:}
    Se comprobó el acceso al servicio HTTPS mediante port forwarding. Como se observa en la Figura \ref{fig:https_desde_isp}, un cliente externo puede acceder correctamente al servidor web interno a través de la dirección pública, demostrando que el port forwarding para el puerto 443 está configurado correctamente.
    
    \begin{figure}[H]
        \centering
        \includegraphics[width=0.8\textwidth]{\pathTests Prueba_https_a_ip_servicios_desde_pc_conectado_a_isp.png}
        \caption{Acceso HTTPS exitoso a la IP pública 192.0.4.203 desde un equipo en la red del ISP.}
        \label{fig:https_desde_isp}
    \end{figure}
	

	\section{Running-Configs}
	\label{sec:runningconfigs}
	
	\subsection{Configuración del Firewall (FW) para CBAC}
	\label{sec:runningconfigs:fw}
	
	\begin{lstlisting}[language=bash, caption=Configuración completa del Firewall FW, label=lst:fw_config]
		Current configuration : 5979 bytes
		!
		! Last configuration change at 11:19:32 UTC Fri Dec 12 2025 by munics
		!
		version 15.4
		service timestamps debug datetime msec
		service timestamps log datetime msec
		no service password-encryption
		!
		hostname fw
		!
		boot-start-marker
		boot-end-marker
		!
		!
		enable secret 5 $1$a48E$tDRpNmlo4YFCSk5yjDWan.
		enable password munics
		!
		aaa new-model
		!
		!
		aaa authentication login default group radius local
		aaa authentication login SSH-LOGIN group radius local-case
		aaa authorization exec default group radius local
		!
		!
		!
		!
		!
		aaa session-id common
		memory-size iomem 15
		!
		!
		!
		!
		!
		!
		!
		!
		!
		!
		!
		!
		!
		!
		no ip domain lookup
		ip domain name munics.pri
		ip inspect name CBAC-FW tcp
		ip inspect name CBAC-FW udp
		ip inspect name CBAC-FW icmp
		ip inspect name CBAC-FW http
		ip inspect name CBAC-FW https
		ip inspect name CBAC-FW ssh
		ip cef
		no ipv6 cef
		!
		multilink bundle-name authenticated
		!
		!
		!
		license udi pid CISCO1941/K9 sn FCZ161592Q9
		!
		!
		username juniorAdmin secret 5 $1$Hhd7$wqazFV5ZhbaQ1ima.Yj5l/
		username admin privilege 15 secret 5 $1$j23q$dXaSI0x7EL24ZPTqWYIT7/
		!
		redundancy
		!
		!
		!
		!
		!
		ip ssh time-out 60
		ip ssh version 2
		!
		!
		!
		!
		!
		!
		!
		!
		!
		!
		interface Embedded-Service-Engine0/0
		no ip address
		shutdown
		!
		interface GigabitEthernet0/0
		no ip address
		duplex auto
		speed auto
		!
		interface GigabitEthernet0/0.2
		encapsulation dot1Q 2
		ip address 10.4.0.2 255.255.255.252
		ip access-group lista_CBAC in
		ip inspect CBAC-FW in
		ip ospf network point-to-point
		!
		interface GigabitEthernet0/0.745
		encapsulation dot1Q 745
		ip address 10.4.245.3 255.255.255.0
		!
		interface GigabitEthernet0/0.746
		encapsulation dot1Q 746
		ip address 10.4.246.1 255.255.255.0
		ip access-group lista_dmz in
		ip inspect CBAC-FW in
		!
		interface GigabitEthernet0/1
		ip address 10.4.0.5 255.255.255.252
		ip access-group FW_INTERNET_IN in
		ip inspect CBAC-FW in
		ip ospf network point-to-point
		duplex auto
		speed auto
		!
		interface Serial0/0/0
		no ip address
		shutdown
		clock rate 2000000
		!
		interface Serial0/0/1
		no ip address
		shutdown
		clock rate 2000000
		!
		router ospf 10
		router-id 3.3.3.3
		passive-interface default
		no passive-interface GigabitEthernet0/0.2
		no passive-interface GigabitEthernet0/1
		network 10.4.0.0 0.0.255.255 area 0
		!
		ip forward-protocol nd
		!
		no ip http server
		no ip http secure-server
		!
		ip route 192.0.0.1 255.255.255.255 10.4.0.6
		!
		ip access-list extended FW_INTERNET_IN
		remark Internet a Servicios (HTTPS especifico + ICMP)
		permit icmp any 10.4.246.0 0.0.0.255
		permit tcp any host 10.4.246.38 eq 443
		remark Permitir OSPF para enrutamiento
		permit ospf host 10.4.0.5 host 224.0.0.5
		permit ospf host 10.4.0.5 host 10.4.0.6
		permit ospf host 10.4.0.6 host 224.0.0.5
		permit ospf host 10.4.0.6 host 10.4.0.5
		remark Denegar todo lo demas
		deny   ip any any log
		ip access-list extended lista_CBAC
		remark VLAN16 -> Servicios (DNS, HTTP, HTTPS, ICMP)
		permit udp 10.4.16.0 0.0.0.255 10.4.246.0 0.0.0.255 eq domain
		permit tcp 10.4.16.0 0.0.0.255 10.4.246.0 0.0.0.255 eq www
		permit tcp 10.4.16.0 0.0.0.255 10.4.246.0 0.0.0.255 eq 443
		permit icmp 10.4.16.0 0.0.0.255 10.4.246.0 0.0.0.255
		permit tcp 10.4.16.0 0.0.0.255 10.4.246.0 0.0.0.255 eq domain
		remark VLAN16 -> Internet (HTTP, HTTPS, DNS, ICMP)
		permit tcp 10.4.16.0 0.0.0.255 any eq www
		permit tcp 10.4.16.0 0.0.0.255 any eq 443
		permit udp 10.4.16.0 0.0.0.255 any eq domain
		permit icmp 10.4.16.0 0.0.0.255 any
		remark VLAN17 -> Servicios (DNS, HTTP, HTTPS, ICMP)
		permit udp 10.4.17.0 0.0.0.255 10.4.246.0 0.0.0.255 eq domain
		permit tcp 10.4.17.0 0.0.0.255 10.4.246.0 0.0.0.255 eq www
		permit tcp 10.4.17.0 0.0.0.255 10.4.246.0 0.0.0.255 eq 443
		permit icmp 10.4.17.0 0.0.0.255 10.4.246.0 0.0.0.255
		permit tcp 10.4.17.0 0.0.0.255 10.4.246.0 0.0.0.255 eq domain
		remark VLAN17 -> Internet (HTTP, HTTPS, DNS, ICMP)
		permit tcp 10.4.17.0 0.0.0.255 any eq www
		permit tcp 10.4.17.0 0.0.0.255 any eq 443
		permit udp 10.4.17.0 0.0.0.255 any eq domain
		permit icmp 10.4.17.0 0.0.0.255 any
		remark VLAN18 -> Servicios (DNS, HTTP, HTTPS, ICMP)
		permit udp 10.4.18.0 0.0.0.255 10.4.246.0 0.0.0.255 eq domain
		permit tcp 10.4.18.0 0.0.0.255 10.4.246.0 0.0.0.255 eq www
		permit tcp 10.4.18.0 0.0.0.255 10.4.246.0 0.0.0.255 eq 443
		permit icmp 10.4.18.0 0.0.0.255 10.4.246.0 0.0.0.255
		permit tcp 10.4.18.0 0.0.0.255 10.4.246.0 0.0.0.255 eq domain
		remark VLAN18 -> Internet (HTTP, HTTPS, DNS, ICMP)
		permit tcp 10.4.18.0 0.0.0.255 any eq www
		permit tcp 10.4.18.0 0.0.0.255 any eq 443
		permit udp 10.4.18.0 0.0.0.255 any eq domain
		permit icmp 10.4.18.0 0.0.0.255 any
		remark ADMIN -> SSH anywhere
		permit tcp 10.4.245.0 0.0.0.255 any eq 22
		remark Permitir OSPF para enrutamiento
		permit ospf host 10.4.0.1 host 224.0.0.5
		permit ospf host 10.4.0.1 host 10.4.0.2
		permit ospf host 10.4.0.2 host 224.0.0.5
		permit ospf host 10.4.0.2 host 10.4.0.1
		remark === REGLAS DE RETORNO PARA SESIONES ESTABLECIDAS ===
		permit tcp any any established
		permit udp any any eq domain
		permit icmp any any echo-reply
		permit icmp any any time-exceeded
		permit icmp any any unreachable
		permit icmp any any echo
		remark === DENEGAR TODO LO DEMAS ===
		deny   ip any any log
		ip access-list extended lista_dmz
		permit tcp 10.4.246.0 0.0.0.255 eq www any
		permit tcp 10.4.246.0 0.0.0.255 eq 443 any
		permit udp 10.4.246.0 0.0.0.255 eq domain any
		permit icmp 10.4.246.0 0.0.0.255 any
		permit icmp any 10.4.246.0 0.0.0.255 echo-reply
		permit tcp any any established
		deny   ip any any log
		!
		!
		!
		access-list 1 permit 10.4.245.0 0.0.0.255
		access-list 1 deny   any
		radius-server host 10.4.245.37 auth-port 1812 acct-port 1813 key Bayern_2025
		!
		!
		!
		control-plane
		!
		!
		!
		line con 0
		password munics
		line aux 0
		line 2
		no activation-character
		no exec
		transport preferred none
		transport output lat pad telnet rlogin lapb-ta mop udptn v120 ssh
		stopbits 1
		line vty 0 4
		access-class 1 in
		password munics
		login authentication SSH-LOGIN
		transport input ssh
		line vty 5 15
		access-class 1 in
		login authentication SSH-LOGIN
		transport input ssh
		!
		scheduler allocate 20000 1000
		!
		end
	\end{lstlisting}
	
	\subsection{Configuración del Router Firewall con Zone-Based Firewall (FW-ZBFW)}
	\label{sec:runningconfigs:fw_zbfw}
	
	\begin{lstlisting}[language=bash, caption=Configuración del Firewall con Zone-Based Firewall, label=lst:fw_zbfw_config]
		! Last configuration change at 11:21:40 UTC Thu Dec 11 2025 by munics
		!
		version 15.4
		service timestamps debug datetime msec
		service timestamps log datetime msec
		no service password-encryption
		!
		hostname fw
		!
		boot-start-marker
		boot-end-marker
		!
		!
		enable secret 5 $1$a48E$tDRpNmlo4YFCSk5yjDWan.
		enable password munics
		!
		aaa new-model
		!
		!
		aaa authentication login default group radius local
		aaa authentication login SSH-LOGIN group radius local-case
		aaa authorization exec default group radius local
		!
		!
		!
		!
		!
		aaa session-id common
		memory-size iomem 15
		!
		!
		!
		!
		!
		!
		!
		!
		!
		!
		!
		!
		!
		!
		no ip domain lookup
		ip domain name munics.pri
		ip cef
		no ipv6 cef
		!
		multilink bundle-name authenticated
		!
		!
		!
		license udi pid CISCO1941/K9 sn FCZ161592Q9
		!
		!
		username juniorAdmin secret 5 $1$Hhd7$wqazFV5ZhbaQ1ima.Yj5l/
		username admin privilege 15 secret 5 $1$j23q$dXaSI0x7EL24ZPTqWYIT7/
		!
		redundancy
		!
		!
		!
		!
		!
		ip ssh time-out 60
		ip ssh version 2
		!
		class-map type inspect match-all CM-OSPF
		match access-group name ACL-OSPF
		class-map type inspect match-all CM-OUTSIDE-TO-DMZ
		match access-group name ACL-OUTSIDE-TO-DMZ
		class-map type inspect match-all CM-DMZ-TO-OUTSIDE
		match access-group name ACL-DMZ-TO-OUTSIDE
		class-map type inspect match-all CM-INSIDE-TO-OUTSIDE
		match access-group name ACL-INSIDE-TO-OUTSIDE
		class-map type inspect match-all CM-INSIDE-TO-DMZ
		match access-group name ACL-INSIDE-TO-DMZ
		!
		policy-map type inspect PM-DMZ-TO-OUTSIDE
		class type inspect CM-DMZ-TO-OUTSIDE
		inspect
		class class-default
		drop log
		policy-map type inspect PM-INSIDE-TO-OUTSIDE
		class type inspect CM-INSIDE-TO-OUTSIDE
		inspect
		class type inspect CM-OSPF
		pass
		class class-default
		drop log
		policy-map type inspect PM-OUTSIDE-TO-DMZ
		class type inspect CM-OUTSIDE-TO-DMZ
		inspect
		class class-default
		drop
		policy-map type inspect PM-INSIDE-TO-DMZ
		class type inspect CM-INSIDE-TO-DMZ
		inspect
		class type inspect CM-OSPF
		pass
		class class-default
		drop log
		!
		zone security OUTSIDE
		description Zona no confiable (Internet)
		zone security DMZ
		description Zona desmilitarizada (Servidores Publicos)
		zone security INSIDE
		description Zona de confianza (Usuarios y Admin)
		zone-pair security ZP-IN-OUT source INSIDE destination OUTSIDE
		service-policy type inspect PM-INSIDE-TO-OUTSIDE
		zone-pair security ZP-IN-DMZ source INSIDE destination DMZ
		service-policy type inspect PM-INSIDE-TO-DMZ
		zone-pair security ZP-DMZ-OUT source DMZ destination OUTSIDE
		service-policy type inspect PM-DMZ-TO-OUTSIDE
		zone-pair security ZP-OUT-DMZ source OUTSIDE destination DMZ
		service-policy type inspect PM-OUTSIDE-TO-DMZ
		!
		!
		!
		!
		!
		!
		!
		!
		!
		!
		interface Embedded-Service-Engine0/0
		no ip address
		shutdown
		!
		interface GigabitEthernet0/0
		no ip address
		duplex auto
		speed auto
		!
		interface GigabitEthernet0/0.2
		encapsulation dot1Q 2
		ip address 10.4.0.2 255.255.255.252
		zone-member security INSIDE
		ip ospf network point-to-point
		!
		interface GigabitEthernet0/0.745
		encapsulation dot1Q 745
		ip address 10.4.245.3 255.255.255.0
		!
		interface GigabitEthernet0/0.746
		encapsulation dot1Q 746
		ip address 10.4.246.1 255.255.255.0
		zone-member security DMZ
		!
		interface GigabitEthernet0/1
		ip address 10.4.0.5 255.255.255.252
		zone-member security OUTSIDE
		ip ospf network point-to-point
		duplex auto
		speed auto
		!
		interface Serial0/0/0
		no ip address
		shutdown
		clock rate 2000000
		!
		interface Serial0/0/1
		no ip address
		shutdown
		clock rate 2000000
		!
		router ospf 10
		router-id 3.3.3.3
		passive-interface default
		no passive-interface GigabitEthernet0/0.2
		no passive-interface GigabitEthernet0/1
		network 10.4.0.0 0.0.255.255 area 0
		!
		ip forward-protocol nd
		!
		no ip http server
		no ip http secure-server
		!
		!
		ip access-list extended ACL-DMZ-TO-OUTSIDE
		remark DMZ -> Internet (HTTP/HTTPS/DNS/ICMP)
		permit tcp 10.4.246.0 0.0.0.255 any eq www
		permit tcp 10.4.246.0 0.0.0.255 any eq 443
		permit udp 10.4.246.0 0.0.0.255 any eq domain
		permit icmp 10.4.246.0 0.0.0.255 any
		ip access-list extended ACL-INSIDE-TO-DMZ
		remark VLAN16 -> DMZ (Servicios)
		permit tcp 10.4.16.0 0.0.0.255 10.4.246.0 0.0.0.255 eq www
		permit tcp 10.4.16.0 0.0.0.255 10.4.246.0 0.0.0.255 eq 443
		permit udp 10.4.16.0 0.0.0.255 10.4.246.0 0.0.0.255 eq domain
		permit tcp 10.4.16.0 0.0.0.255 10.4.246.0 0.0.0.255 eq domain
		permit icmp 10.4.16.0 0.0.0.255 10.4.246.0 0.0.0.255
		remark VLAN17 -> DMZ (Servicios)
		permit tcp 10.4.17.0 0.0.0.255 10.4.246.0 0.0.0.255 eq www
		permit tcp 10.4.17.0 0.0.0.255 10.4.246.0 0.0.0.255 eq 443
		permit udp 10.4.17.0 0.0.0.255 10.4.246.0 0.0.0.255 eq domain
		permit tcp 10.4.17.0 0.0.0.255 10.4.246.0 0.0.0.255 eq domain
		permit icmp 10.4.17.0 0.0.0.255 10.4.246.0 0.0.0.255
		remark VLAN18 -> DMZ (Servicios)
		permit tcp 10.4.18.0 0.0.0.255 10.4.246.0 0.0.0.255 eq www
		permit tcp 10.4.18.0 0.0.0.255 10.4.246.0 0.0.0.255 eq 443
		permit udp 10.4.18.0 0.0.0.255 10.4.246.0 0.0.0.255 eq domain
		permit tcp 10.4.18.0 0.0.0.255 10.4.246.0 0.0.0.255 eq domain
		permit icmp 10.4.18.0 0.0.0.255 10.4.246.0 0.0.0.255
		remark VLAN16 -> DMZ (HTTP/HTTPS/DNS/ICMP)
		remark VLAN17 -> DMZ
		remark VLAN18 -> DMZ
		ip access-list extended ACL-INSIDE-TO-OUTSIDE
		remark VLAN16 -> Internet
		permit tcp 10.4.16.0 0.0.0.255 any eq www
		permit tcp 10.4.16.0 0.0.0.255 any eq 443
		permit udp 10.4.16.0 0.0.0.255 any eq domain
		permit tcp 10.4.16.0 0.0.0.255 any eq domain
		permit icmp 10.4.16.0 0.0.0.255 any
		remark VLAN17 -> Internet
		permit tcp 10.4.17.0 0.0.0.255 any eq www
		permit tcp 10.4.17.0 0.0.0.255 any eq 443
		permit udp 10.4.17.0 0.0.0.255 any eq domain
		permit tcp 10.4.17.0 0.0.0.255 any eq domain
		permit icmp 10.4.17.0 0.0.0.255 any
		remark VLAN18 -> Internet
		permit tcp 10.4.18.0 0.0.0.255 any eq www
		permit tcp 10.4.18.0 0.0.0.255 any eq 443
		permit udp 10.4.18.0 0.0.0.255 any eq domain
		permit tcp 10.4.18.0 0.0.0.255 any eq domain
		permit icmp 10.4.18.0 0.0.0.255 any
		ip access-list extended ACL-OSPF
		remark OSPF entre todos los routers
		permit ospf host 10.4.0.1 host 224.0.0.5
		permit ospf host 10.4.0.1 host 10.4.0.2
		permit ospf host 10.4.0.2 host 224.0.0.5
		permit ospf host 10.4.0.2 host 10.4.0.1
		permit ospf host 10.4.0.5 host 224.0.0.5
		permit ospf host 10.4.0.5 host 10.4.0.6
		permit ospf host 10.4.0.6 host 224.0.0.5
		permit ospf host 10.4.0.6 host 10.4.0.5
		ip access-list extended ACL-OUTSIDE-TO-DMZ
		remark Internet -> Servidor HTTPS en DMZ + Ping
		permit icmp any host 10.4.246.38
		permit tcp any host 10.4.246.38 eq 443
		!
		!
		!
		access-list 1 permit 10.4.245.0 0.0.0.255
		access-list 1 deny   any
		radius-server host 10.4.245.37 auth-port 1812 acct-port 1813 key Bayern_2025
		!
		!
		!
		control-plane
		!
		!
		!
		line con 0
		password munics
		line aux 0
		line 2
		no activation-character
		no exec
		transport preferred none
		transport output lat pad telnet rlogin lapb-ta mop udptn v120 ssh
		stopbits 1
		line vty 0 4
		access-class 1 in
		password munics
		login authentication SSH-LOGIN
		transport input ssh
		line vty 5 15
		access-class 1 in
		login authentication SSH-LOGIN
		transport input ssh
		!
		scheduler allocate 20000 1000
		!
		end
	\end{lstlisting}
	
	\subsection{Configuración del Router CPE}
	\label{sec:runningconfigs:cpe}
	
	\begin{lstlisting}[language=bash, caption=Configuración completa del Router CPE, label=lst:cpe_config]
		Current configuration : 4083 bytes
		!
		! Last configuration change at 06:02:57 UTC Thu Feb 7 2036 by munics
		!
		version 15.4
		service timestamps debug datetime msec
		service timestamps log datetime msec
		no service password-encryption
		!
		hostname CPE
		!
		boot-start-marker
		boot-end-marker
		!
		!
		enable secret 5 $1$GdZU$wjEfpZdqvFc/slyD.nmT4.
		enable password munics
		!
		aaa new-model
		!
		!
		aaa authentication login default group radius local enable
		aaa authentication login SSH-LOGIN group radius local-case
		aaa authentication login ssh-login group radius local-case
		aaa authorization exec default group radius local
		!
		!
		!
		!
		!
		aaa session-id common
		memory-size iomem 15
		!
		!
		!
		!
		!
		!
		!
		!
		!
		!
		!
		!
		!
		!
		no ip domain lookup
		ip domain name munics.pri
		ip cef
		no ipv6 cef
		!
		multilink bundle-name authenticated
		!
		!
		!
		license udi pid CISCO1941/K9 sn FCZ1520C07N
		!
		!
		username juniorAdmin secret 5 $1$kRPU$yy5tuKlqErLUXYYvtwjOZ.
		username admin privilege 15 secret 5 $1$vHGy$Bg1OGfhpYB6RjnZFvWVgC1
		!
		redundancy
		!
		!
		!
		!
		!
		ip ssh time-out 60
		ip ssh version 2
		!
		!
		!
		!
		!
		!
		!
		!
		!
		!
		interface Embedded-Service-Engine0/0
		no ip address
		shutdown
		!
		interface GigabitEthernet0/0
		no ip address
		duplex auto
		speed auto
		!
		interface GigabitEthernet0/0.3
		encapsulation dot1Q 3
		ip address 10.4.0.6 255.255.255.252
		ip nat inside
		ip virtual-reassembly in
		ip ospf network point-to-point
		!
		interface GigabitEthernet0/0.745
		encapsulation dot1Q 745
		ip address 10.4.245.4 255.255.255.0
		!
		interface GigabitEthernet0/1
		ip address 192.0.4.1 255.255.255.0
		ip access-group 10 in
		ip nat outside
		ip virtual-reassembly in
		duplex auto
		speed auto
		!
		interface Serial0/0/0
		no ip address
		shutdown
		clock rate 2000000
		!
		interface Serial0/0/1
		no ip address
		shutdown
		clock rate 2000000
		!
		router ospf 10
		router-id 4.4.4.4
		passive-interface default
		no passive-interface GigabitEthernet0/0.3
		network 10.4.0.0 0.0.255.255 area 0
		default-information originate
		!
		ip forward-protocol nd
		!
		no ip http server
		no ip http secure-server
		!
		ip nat pool POOL_ALUMNOS 192.0.4.200 192.0.4.200 netmask 255.255.255.0
		ip nat pool POOL_PDI 192.0.4.201 192.0.4.201 netmask 255.255.255.0
		ip nat pool POOL_PAS 192.0.4.202 192.0.4.202 netmask 255.255.255.0
		ip nat inside source list ACL_NAT_ALUMNOS pool POOL_ALUMNOS overload
		ip nat inside source list ACL_NAT_PAS pool POOL_PAS overload
		ip nat inside source list ACL_NAT_PDI pool POOL_PDI overload
		ip nat inside source static tcp 10.4.246.38 80 192.0.4.203 80 extendable
		ip nat inside source static tcp 10.4.246.38 443 192.0.4.203 443 extendable
		ip route 0.0.0.0 0.0.0.0 192.0.4.2
		ip route 192.0.0.0 255.255.255.0 192.0.4.2
		!
		ip access-list standard ACL_NAT_ALUMNOS
		permit 10.4.16.0 0.0.0.255
		ip access-list standard ACL_NAT_PAS
		permit 10.4.18.0 0.0.0.255
		ip access-list standard ACL_NAT_PDI
		permit 10.4.17.0 0.0.0.255
		!
		!
		!
		access-list 1 permit 10.4.245.0 0.0.0.255
		access-list 1 deny   any
		access-list 10 deny   255.255.255.255
		access-list 10 deny   0.0.0.0 0.255.255.255
		access-list 10 deny   10.0.0.0 0.255.255.255
		access-list 10 deny   100.64.0.0 0.63.255.255
		access-list 10 deny   127.0.0.0 0.255.255.255
		access-list 10 deny   169.254.0.0 0.0.255.255
		access-list 10 deny   172.16.0.0 0.15.255.255
		access-list 10 deny   192.88.99.0 0.0.0.255
		access-list 10 deny   192.168.0.0 0.0.255.255
		access-list 10 deny   198.18.0.0 0.1.255.255
		access-list 10 deny   198.51.100.0 0.0.0.255
		access-list 10 deny   203.0.113.0 0.0.0.255
		access-list 10 deny   224.0.0.0 15.255.255.255
		access-list 10 deny   240.0.0.0 15.255.255.255
		access-list 10 permit any
		radius-server host 10.4.245.37 auth-port 1812 acct-port 1813
		radius-server key Bayern_2025
		!
		radius server DL-SW-RADIUS
		!
		radius server CPE-RADIUS
		!
		!
		!
		control-plane
		!
		!
		!
		line con 0
		password munics
		line aux 0
		line 2
		no activation-character
		no exec
		transport preferred none
		transport output lat pad telnet rlogin lapb-ta mop udptn v120 ssh
		stopbits 1
		line vty 0 4
		access-class 1 in
		password munics
		login authentication ssh-login
		transport input ssh
		line vty 5 15
		access-class 1 in
		login authentication ssh-login
		transport input ssh
		!
		scheduler allocate 20000 1000
		!
		end
	\end{lstlisting}
	
	\subsection{Configuración del Switch de Distribución (DL-SW1)}
	\label{sec:runningconfigs:dlsw1}
	
	\begin{lstlisting}[language=bash, caption=Configuración completa del Switch DL-SW1, label=lst:dl_sw1_config]
		Current configuration : 5321 bytes
		!
		version 12.2
		no service pad
		service timestamps debug datetime msec
		service timestamps log datetime msec
		no service password-encryption
		!
		hostname DL-SW1
		!
		boot-start-marker
		boot-end-marker
		!
		enable secret 5 $1$h4mP$R/iJAmdAFuSOG0hT31MoL/
		enable password munics
		!
		username juniorAdmin secret 5 $1$Kv2T$zH/V14q21ulpWg.CkcYOX.
		username admin privilege 15 secret 5 $1$/Bku$u1UXOu1a5D3u8XSooyC0P1
		!
		!
		aaa new-model
		!
		!
		aaa group server radius RADIUS-GROUP
		!
		aaa authentication login default group radius local
		aaa authentication login SSH-LOGIN group radius local-case
		aaa authorization exec default group radius local
		!
		!
		!
		aaa session-id common
		system mtu routing 1500
		ip routing
		no ip domain-lookup
		ip domain-name munics.pri
		!
		!
		!
		!
		crypto pki trustpoint TP-self-signed-3123997184
		enrollment selfsigned
		subject-name cn=IOS-Self-Signed-Certificate-3123997184
		revocation-check none
		rsakeypair TP-self-signed-3123997184
		!
		!
		!
		!
		!
		!
		spanning-tree mode pvst
		spanning-tree extend system-id
		spanning-tree vlan 16-18,745 priority 24576
		!
		vlan internal allocation policy ascending
		!
		ip ssh time-out 60
		ip ssh authentication-retries 2
		ip ssh version 2
		!
		!
		!
		interface FastEthernet0/1
		switchport mode access
		switchport port-security maximum 10
		switchport port-security
		switchport port-security mac-address sticky
		!
		interface FastEthernet0/2
		!
		interface FastEthernet0/3
		!
		interface FastEthernet0/4
		!
		interface FastEthernet0/5
		!
		interface FastEthernet0/6
		!
		interface FastEthernet0/7
		!
		interface FastEthernet0/8
		!
		interface FastEthernet0/9
		!
		interface FastEthernet0/10
		!
		interface FastEthernet0/11
		!
		interface FastEthernet0/12
		switchport trunk encapsulation dot1q
		switchport trunk allowed vlan 16-18,745
		switchport mode trunk
		!
		interface FastEthernet0/13
		switchport trunk encapsulation dot1q
		switchport trunk allowed vlan 2,745,746
		switchport mode trunk
		!
		interface FastEthernet0/14
		switchport access vlan 3
		switchport mode access
		!
		interface FastEthernet0/15
		switchport trunk encapsulation dot1q
		switchport trunk allowed vlan 3,745
		switchport mode trunk
		!
		interface FastEthernet0/16
		switchport access vlan 4
		switchport mode access
		!
		interface FastEthernet0/17
		switchport trunk encapsulation dot1q
		switchport trunk allowed vlan 4,745
		switchport mode trunk
		!
		interface FastEthernet0/18
		!
		interface FastEthernet0/19
		!
		interface FastEthernet0/20
		!
		interface FastEthernet0/21
		!
		interface FastEthernet0/22
		!
		interface FastEthernet0/23
		!
		interface FastEthernet0/24
		switchport trunk encapsulation dot1q
		switchport trunk allowed vlan 745,746
		switchport mode trunk
		!
		interface GigabitEthernet0/1
		!
		interface GigabitEthernet0/2
		!
		interface Vlan1
		no ip address
		!
		interface Vlan2
		ip address 10.4.0.1 255.255.255.252
		ip ospf network point-to-point
		!
		interface Vlan16
		ip address 10.4.16.1 255.255.255.0
		ip access-group VLAN16 in
		ip helper-address 10.4.245.100
		!
		interface Vlan17
		ip address 10.4.17.1 255.255.255.0
		ip access-group VLAN17 in
		ip helper-address 10.4.245.100
		!
		interface Vlan18
		ip address 10.4.18.1 255.255.255.0
		ip access-group VLAN18 in
		ip helper-address 10.4.245.100
		!
		interface Vlan745
		ip address 10.4.245.2 255.255.255.0
		!
		router ospf 10
		router-id 2.2.2.2
		log-adjacency-changes
		passive-interface default
		no passive-interface Vlan2
		network 10.4.0.0 0.0.255.255 area 0
		!
		ip classless
		ip route 0.0.0.0 0.0.0.0 10.4.0.2
		ip http server
		ip http secure-server
		!
		!
		ip access-list extended VLAN16
		permit udp any any eq bootps
		permit udp any any eq bootpc
		deny   ip 10.4.16.0 0.0.0.255 10.4.17.0 0.0.0.255
		deny   ip 10.4.16.0 0.0.0.255 10.4.18.0 0.0.0.255
		deny   ip 10.4.16.0 0.0.0.255 10.4.245.0 0.0.0.255
		deny   ip 10.4.16.0 0.0.0.255 10.4.0.0 0.0.0.3
		permit tcp 10.4.16.0 0.0.0.255 any eq www
		permit tcp 10.4.16.0 0.0.0.255 any eq 443
		permit udp 10.4.16.0 0.0.0.255 any eq domain
		permit tcp 10.4.16.0 0.0.0.255 any eq domain
		permit icmp 10.4.16.0 0.0.0.255 any
		ip access-list extended VLAN17
		permit udp any any eq bootps
		permit udp any any eq bootpc
		deny   ip 10.4.17.0 0.0.0.255 10.4.16.0 0.0.0.255
		deny   ip 10.4.17.0 0.0.0.255 10.4.18.0 0.0.0.255
		deny   ip 10.4.17.0 0.0.0.255 10.4.245.0 0.0.0.255
		deny   ip 10.4.17.0 0.0.0.255 10.4.0.0 0.0.0.3
		permit tcp 10.4.17.0 0.0.0.255 any eq www
		permit tcp 10.4.17.0 0.0.0.255 any eq 443
		permit udp 10.4.17.0 0.0.0.255 any eq domain
		permit tcp 10.4.17.0 0.0.0.255 any eq domain
		permit icmp 10.4.17.0 0.0.0.255 any
		ip access-list extended VLAN18
		permit udp any any eq bootps
		permit udp any any eq bootpc
		deny   ip 10.4.18.0 0.0.0.255 10.4.16.0 0.0.0.255
		deny   ip 10.4.18.0 0.0.0.255 10.4.17.0 0.0.0.255
		deny   ip 10.4.18.0 0.0.0.255 10.4.245.0 0.0.0.255
		deny   ip 10.4.18.0 0.0.0.255 10.4.0.0 0.0.0.3
		permit tcp 10.4.18.0 0.0.0.255 any eq www
		permit tcp 10.4.18.0 0.0.0.255 any eq 443
		permit udp 10.4.18.0 0.0.0.255 any eq domain
		permit tcp 10.4.18.0 0.0.0.255 any eq domain
		permit icmp 10.4.18.0 0.0.0.255 any
		!
		ip sla enable reaction-alerts
		access-list 1 permit 10.4.245.0 0.0.0.255
		access-list 1 deny   any
		!
		radius-server host 10.4.245.37 auth-port 1812 acct-port 1813 key Bayern_2025
		!
		!
		line con 0
		password munics
		line vty 0 4
		access-class 1 in
		password munics
		login authentication SSH-LOGIN
		transport input ssh
		line vty 5 15
		access-class 1 in
		login authentication SSH-LOGIN
		transport input ssh
		!
		end
	\end{lstlisting}
	
	
	\subsection{Configuración del Router ISP}
	\label{sec:runningconfigs:isp}
	
	\begin{lstlisting}[language=bash, caption=Configuración del Router ISP, label=lst:isp_config]
		Current configuration : 2222 bytes
		!
		! Last configuration change at 11:08:22 UTC Thu Mar 20 2025 by munics
		!
		version 15.4
		service timestamps debug datetime msec
		service timestamps log datetime msec
		no service password-encryption
		!
		hostname ISP
		!
		boot-start-marker
		boot-end-marker
		!
		!
		enable secret 5 $1$Y1Sr$TxKfVgh6.IjLVIx9YtRSN/
		enable password munics
		!
		aaa new-model
		!
		!
		aaa authentication login default group radius local
		aaa authentication login SSH-LOGIN group radius local-case
		aaa authorization exec default group radius local
		!
		!
		!
		!
		!
		aaa session-id common
		memory-size iomem 15
		!
		!
		!
		!
		!
		!
		!
		!
		!
		!
		!
		!
		!
		!
		no ip domain lookup
		ip domain name acme.pri
		ip cef
		no ipv6 cef
		!
		multilink bundle-name authenticated
		!
		!
		!
		license udi pid CISCO1941/K9 sn FCZ151592DL
		!
		!
		username juniorAdmin secret 5 $1$iz50$zTkxSpdIv3CtwygyBrqWj0
		username admin privilege 15 secret 5 $1$XEbC$I3tqFuyyCK.WqIu1Fx1mf1
		!
		redundancy
		!
		!
		!
		!
		!
		ip ssh time-out 60
		ip ssh version 2
		!
		!
		!
		!
		!
		!
		!
		!
		!
		!
		interface Embedded-Service-Engine0/0
		no ip address
		shutdown
		!
		interface GigabitEthernet0/0
		no ip address
		duplex auto
		speed auto
		!
		interface GigabitEthernet0/0.4
		encapsulation dot1Q 4
		ip address 192.0.4.2 255.255.255.0
		!
		interface GigabitEthernet0/0.745
		description VLAN-Pod4-adm
		encapsulation dot1Q 745
		ip address 10.4.245.5 255.255.255.0
		!
		interface GigabitEthernet0/1
		ip address 192.0.0.4 255.255.255.0
		duplex auto
		speed auto
		!
		interface Serial0/0/0
		no ip address
		shutdown
		clock rate 2000000
		!
		interface Serial0/0/1
		no ip address
		shutdown
		clock rate 2000000
		!
		ip forward-protocol nd
		!
		no ip http server
		ip http secure-server
		!
		ip route 10.4.0.0 255.255.0.0 192.0.4.1
		!
		!
		!
		access-list 1 permit 10.4.245.0 0.0.0.255
		access-list 1 deny   any
		radius-server host 10.4.245.37 auth-port 1812 acct-port 1813 key Bayern_2025
		!
		!
		!
		control-plane
		!
		!
		!
		line con 0
		password munics
		line aux 0
		line 2
		no activation-character
		no exec
		transport preferred none
		transport output lat pad telnet rlogin lapb-ta mop udptn v120 ssh
		stopbits 1
		line vty 0 4
		access-class 1 in
		password munics
		login authentication SSH-LOGIN
		transport input ssh
		line vty 5 15
		access-class 1 in
		login authentication SSH-LOGIN
		transport input ssh
		!
		scheduler allocate 20000 1000
		!
		end
	\end{lstlisting}
	
\end{document}
