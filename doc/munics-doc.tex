\documentclass{ltxdockit}
\usepackage{btxdockit}
\usepackage[utf8]{inputenc}
\usepackage[spanish]{babel}
\usepackage[strict]{csquotes}
\usepackage{tabularx}
\usepackage{longtable}
\usepackage{booktabs}
\usepackage{shortvrb}
\usepackage{pifont}
\usepackage{graphicx}
\usepackage{hyperref}
\usepackage{listings}
\usepackage{xcolor}
\usepackage{float}
\newcommand{\pathServ}{images/servicios/}
\newcommand{\pathAcls}{images/acls/}
\newcommand{\pathNat}{images/nat/}
\newcommand{\pathZbfw}{images/zbfw/}
\newcommand{\pathTests}{images/pruebas/}
\newcommand{\pathRoot}{images/}



% Definición de colores para código
\definecolor{codegreen}{rgb}{0,0.6,0}
\definecolor{codegray}{rgb}{0.5,0.5,0.5}
\definecolor{codepurple}{rgb}{0.58,0,0.82}
\definecolor{backcolour}{rgb}{0.95,0.95,0.92}

\lstdefinestyle{bashstyle}{
	backgroundcolor=\color{backcolour},   
	commentstyle=\color{codegreen},
	keywordstyle=\color{magenta},
	numberstyle=\tiny\color{codegray},
	stringstyle=\color{codepurple},
	basicstyle=\ttfamily\footnotesize,
	breakatwhitespace=false,         
	breaklines=true,                 
	captionpos=b,                    
	keepspaces=true,                 
	numbers=left,                    
	numbersep=5pt,                  
	showspaces=false,                
	showstringspaces=false,
	showtabs=false,                  
	tabsize=2,
	frame=single,
	language=bash
}

\lstset{style=bashstyle}

\newcommand*{\munics}{\emph{MUniCS Master's Thesis}\xspace}

\titlepage{%
	title={SEGURIDAD DE REDES},
	subtitle={Laboratorio 8 - Seguridad Perimetral},
	url={},
	author={Estela Pillo González \\ Orlando J. Garcés Casal \\ Simón Noya Dominguez \\ Alvaro Cainzos Urtiaga \\ Brais Gómez Espiñeira},
	email={estela.pgonzalez@udc.es \\ o.garces@udc.es \\ simon.noyad@udc.es \\ alvaro.cainzos@udc.es \\ brais.gomez2@udc.es},
	revision={1.0},
	date={\today}}

\hypersetup{%
	pdftitle={Documentación de Laboratorio 8 - Seguridad Perimetral},
	pdfsubject={Laboratorio 8 - Seguridad Perimetral},
	pdfauthor={Nombre del Estudiante},
	pdfkeywords={redes, seguridad, firewall, ACL, NAT, perimetral}}

\begin{document}
	
	\printtitlepage
	\tableofcontents
	
	\section{Introducción}
	\label{sec:intro}
	
	A continuación se presenta la memoria de este laboratorio, donde se explica cómo se desarrolló toda la práctica paso a paso. 
	En este documento se describe la configuración realizada en cada parte del entorno, incluyendo la puesta en marcha de los
	servidores, la aplicación de filtros y reglas de seguridad, la configuración del firewall con distintas técnicas, 
	la implementación de NAT y port forwarding, así como las pruebas finales de conectividad para verificar que todo 
	funciona correctamente.
	
	\section{Laboratorio 8: Seguridad Perimetral}
	\label{sec:lab8}
	
	\subsection{Parte 1: Despliegue de Servicios}
	\label{sec:lab8:parte1}
	
	En esta primera parte del laboratorio se preparó la infraestructura básica necesaria para que las distintas VLAN del entorno pudieran acceder a servicios internos comunes. 
	Para ello, en la VLAN de servicios se desplegaron tres componentes principales: un servidor web accesible tanto por HTTP como por HTTPS, y un servidor DNS encargado de resolver los nombres del dominio interno. 

	\subsubsection{Configuración del servidor HTTP y HTTPS}
	\label{sec:lab8:servidores:http}
	
	Se decidió configurar el servidor web sobre \texttt{nginx}.\\
	Para habilitar HTTPS fue necesaria la creación de un certificado SSL autofirmado por nginx. Este procedimiento se muestra en la Figura \ref{fig:nginx1}:
	
	\begin{figure}[H]
		\centering
		\includegraphics[width=\textwidth]{\pathServ nginx_config_1.png}
		\caption{Generación y firma del certificado SSL.}
		\label{fig:nginx1}
	\end{figure}
	
	A continuación creamos y configuramos las carpetas de la web. En la Figura \ref{fig:nginx2} se muestran los distintos comandos utilizados para la creación y configuración de la carpeta de la web. En conjunto se realiza lo siguiente:
	\begin{itemize}
		\item Creación de la carpeta del sitio web.
		\item Se proporcionan permisos para que nginx sea capaz de acceder.
		\item Creación del archivo de configuración de la web.
		\item Link de los archivos a sites-enabled.
		\item Comprobación de la configuración de nginx.
	\end{itemize}
	
	\begin{figure}[H]
		\centering
		\includegraphics[width=\textwidth]{\pathServ nginx_config_2.png}
		\caption{Creación de archivos necesarios para el servidor.}
		\label{fig:nginx2}
	\end{figure}
	
	Es necesario configurar el servidor web para manejar el tráfico HTTP y HTTPS. En la Figura \ref{fig:nginx_http} se muestra la configuración para manejar el tráfico HTTP.
	
	\begin{figure}[H]
		\centering
		\includegraphics[width=\textwidth]{\pathServ nginx_config_http.png}
		\caption{Configuración nginx HTTP.}
		\label{fig:nginx_http}
	\end{figure}
	

	El servicio HTTPS se configuró tal como muestra la Figura \ref{fig:nginx_https}. 
	
	\begin{figure}[H]
		\centering
		\includegraphics[width=\textwidth]{\pathServ nginx_config_https.png}
		\caption{Configuración nginx HTTPS.}
		\label{fig:nginx_https}
	\end{figure}
	
	El contenido publicado consistió en una página HTML sencilla, mostrada en la Figura \ref{fig:nginx_html}:
	
	\begin{figure}[H]
		\centering
		\includegraphics[width=\textwidth]{\pathServ nginx_config_indexhtml.png}
		\caption{Contenido del archivo \texttt{index.html}}
		\label{fig:nginx_html}
	\end{figure}
	
	Finalmente, tal como se muestra en la Figura \ref{fig:nginx_acceso}, se comprueba que los clientes de la red podían acceder correctamente al servidor web.
	
	\begin{figure}[H]
		\centering
		\includegraphics[width=\textwidth]{\pathServ nginx_check_works.png}
		\caption{Acceso al servidor web.}
		\label{fig:nginx_acceso}
	\end{figure}
	
	\subsubsection{Configuración del Servidor DNS}
	\label{sec:lab8:servidores:dns}
	
	Para proporcionar resolución de nombres dentro de la red, se configuró un servidor DNS mediante \texttt{bind9}.
	
	Para el funcionamiento del servidor DNS se requirieron los siguientes pasos:
	\begin{itemize}
		\item Configurar las opciones principales del servidor DNS (\ref{fig:dns1}).
		\item Definir la zona DNS que maneja el servidor (\ref{fig:dns})).
		\item Creación de archivo de zona directa (\ref{fig:dns3})
		\end{itemize}
	
	\begin{figure}[H]
		\centering
		\includegraphics[width=\textwidth]{\pathServ DNS_1.png}
		\caption{Configuración inicial del servidor DNS.}
		\label{fig:dns1}
		
	\end{figure}
	
	\begin{figure}[H]
		\centering
		\includegraphics[width=\textwidth]{\pathServ DNS_2.png}
		\caption{Definición de la zona DNS interna.}
		\label{fig:dns2}
	\end{figure}
	
	\begin{figure}[H]
		\centering
		\includegraphics[width=\textwidth]{\pathServ DNS_3.png}
		\caption{Creación y definición de archivos de zona.}
		\label{fig:dns3}
	\end{figure}
	
	En la Figura \ref{fig:dns4} se comprueba todo lo implementado y el estado actual del servicio.
	\begin{figure}[H]
		\centering
		\includegraphics[width=\textwidth]{\pathServ DNS_4.png}
		\caption{Comprobación del funcionamiento del servicio.}
		\label{fig:dns4}
	\end{figure}
	
	En la Figura \ref{fig:dns5} de muestra una prueba de la resolución DNS y acceso mediante HTTP.
	\begin{figure}[H]
		\centering
		\includegraphics[width=\textwidth]{\pathServ DNS_5.png}
		\caption{Comprobación del funcionamiento del servicio, parte 2.}
		\label{fig:dns5}
	\end{figure}
	
	Por último, se comprueba el funcionamiento del servicio HTTPS en la Figura \ref{fig:dns6}.
	\begin{figure}[H]
		\centering
		\includegraphics[width=\textwidth]{\pathServ DNS_6.png}
		\caption{Comprobación del funcionamiento del servicio, parte 3.}
		\label{fig:dns6}
	\end{figure}
	
	\subsection{Parte 2: Configuración de Filtros en los Firewall}
	\label{sec:lab8:parte2}
	
	\subsubsection{Filtrado Estático de Paquetes}
	\label{sec:lab8:filtrado-estatico}
	
	Descripción de la ACL estándar implementada para bloquear direcciones IP no válidas o peligrosas:
	\begin{itemize}
		\item Direcciones privadas.
		\item Direcciones de enlace local.
		\item Direcciones multicast y broadcast.
		\item Loopback (127.0.0.0/8).
		\item Otras direcciones consideradas peligrosas (justificar).
	\end{itemize}
	
	Incluir aquí la configuración de la ACL estándar en el firewall.
	
	\subsubsection{Configuración de CBAC en FW}
	\label{sec:lab8:cbac}
	
	Implementación de la política de seguridad definida en la Tabla 1 mediante CBAC. Consideraciones:
	\begin{itemize}
		\item Tráfico bidireccional.
		\item Protocolos de enrutamiento (OSPF, etc.).
		\item Plano de gestión (SSH desde VLAN de administración).
		\item Acceso desde equipos de salto.
	\end{itemize}
	
	Incluir configuración de las ACLs extendidas y comandos CBAC aplicados.
	
	\subsubsection{Configuración de Zone-Based Firewall en FW}
	\label{sec:lab8:zbfw}
	
	Implementación alternativa de la política de seguridad mediante ZBFW:
	\begin{itemize}
		\item Definición de zonas (inside, outside, DMZ, etc.).
		\item Políticas entre zonas.
		\item Comparación CBAC vs ZBFW (ventajas/desventajas en esta topología).
	\end{itemize}
	
	
	\begin{figure}[H]
		\centering
		\includegraphics[width=\textwidth]{\pathZbfw class maps.png}
		\caption{Class-maps para inspección de tráfico en el Zone-Based Firewall.}
		\label{fig:zbfw_class_maps}
	\end{figure}
	
	\begin{figure}[H]
		\centering
		\includegraphics[width=\textwidth]{\pathZbfw policy map inside to outside .png}
		\caption{Policy-map para tráfico \emph{inside} $\rightarrow$ \emph{outside}.}
		\label{fig:zbfw_policy_inside_outside}
	\end{figure}
	
	\begin{figure}[H]
		\centering
		\includegraphics[width=\textwidth]{\pathZbfw policy map inside to dmz.png}
		\caption{Policy-map para tráfico \emph{inside} $\rightarrow$ DMZ.}
		\label{fig:zbfw_policy_inside_dmz}
	\end{figure}
	
	\begin{figure}[H]
		\centering
		\includegraphics[width=\textwidth]{\pathZbfw policy map dmz to outside.png}
		\caption{Policy-map para tráfico DMZ $\rightarrow$ \emph{outside}.}
		\label{fig:zbfw_policy_dmz_outside}
	\end{figure}
	
	\begin{figure}[H]
		\centering
		\includegraphics[width=\textwidth]{\pathZbfw policy map outside to dmz.png}
		\caption{Policy-map para tráfico \emph{outside} $\rightarrow$ DMZ.}
		\label{fig:zbfw_policy_outside_dmz}
	\end{figure}
	
	\begin{figure}[H]
		\centering
		\includegraphics[width=\textwidth]{\pathZbfw zone pair in-out.png}
		\caption{Zone-pair entre las zonas \emph{inside} y \emph{outside}.}
		\label{fig:zbfw_zonepair_in_out}
	\end{figure}
	
	\begin{figure}[H]
		\centering
		\includegraphics[width=\textwidth]{\pathZbfw zone pair in-dmz.png}
		\caption{Zone-pair entre \emph{inside} y DMZ.}
		\label{fig:zbfw_zonepair_in_dmz}
	\end{figure}
	
	\begin{figure}[H]
		\centering
		\includegraphics[width=\textwidth]{\pathZbfw zone pair dmz-out.png}
		\caption{Zone-pair entre DMZ y \emph{outside}.}
		\label{fig:zbfw_zonepair_dmz_out}
	\end{figure}
	
	\begin{figure}[H]
		\centering
		\includegraphics[width=\textwidth]{\pathZbfw zone pair out-dmz.png}
		\caption{Zone-pair entre \emph{outside} y DMZ.}
		\label{fig:zbfw_zonepair_out_dmz}
	\end{figure}
	
	\begin{figure}[H]
		\centering
		\includegraphics[width=\textwidth]{\pathZbfw zone securities.png}
		\caption{Resumen de las zonas de seguridad configuradas en el firewall.}
		\label{fig:zbfw_zone_securities}
	\end{figure}
	
	\begin{figure}[H]
		\centering
		\includegraphics[width=\textwidth]{\pathZbfw SSH_ZBFW.png}
		\caption{Comprobación de acceso SSH protegido por el Zone-Based Firewall.}
		\label{fig:zbfw_ssh}
	\end{figure}
	
	\begin{figure}[H]
		\centering
		\includegraphics[width=\textwidth]{\pathZbfw asignacion zonas a interfaces.png}
		\caption{Asignación de interfaces físicas a zonas de seguridad.}
		\label{fig:zbfw_zonas_interfaces}
	\end{figure}
	
	\subsubsection{Filtrado en DL-SW1}
	\label{sec:lab8:filtrado-switch}
	
	Configuración de control de tráfico entre VLANs en el switch de distribución:
	\begin{itemize}
		\item Opciones de filtrado disponibles en el switch.
		\item Justificación de la opción elegida.
		\item Configuración aplicada (ACLs en capa 3, VACLs, etc.).
	\end{itemize}
	
	\subsection{Parte 3: Configuración de NAT}
	\label{sec:lab8:parte3}
	
	En esta parte del laboratorio se configuró la traducción de direcciones de red (NAT) en el CPE con el objetivo de permitir
	que las distintas VLAN de la organización pudieran acceder a Internet utilizando direcciones IP públicas del rango asignado
	al Pod. Además, se configuró port forwarding para publicar los servicios internos (HTTP y HTTPS) hacia el exterior.
	
	
	\subsubsection{PAT dinámico para tráfico saliente}
	\label{sec:lab8:pat}

	Para implementar este comportamiento, el primer paso fue crear un pool de direcciones para cada VLAN. Estos pools permiten que se asignen direcciones dentro de un rango determinado para cada una de las VLANs. Se muestran en la imagen  TALTAL junto con las ACL necesarias para identificar el tráfico origen de cada red. \\
	
	%%%añadir imagen
	
	Las ACL generadas permiten clasificar el tráfico según la VLAN de origen, mientras que los pools determinan qué dirección pública se asignará a cada una de ellas. \\
	
	El siguiente paso consistió en asociar cada ACL con su pool correspondiente y activar PAT mediante la opción
	\texttt{overload}. La figura TALTAL muestra este proceso: \\
	
	%%añadir imagen
	
	
	A continuación, en la imagen \ref{fig:nat_interfaces} se configuran las interfaces como inside y outside. La interfaz GigabitEthernet0/0.3 se configura como \texttt{inside}, mientras que la interfaz GigabitEthernet0/1 se configura como \texttt{outside}.
	
	\begin{figure}[H]
		\centering
		\includegraphics[width=\textwidth]{\pathNat asignacion nat a interfaces.png}
		\caption{Asignación de NAT (inside/outside) a las interfaces.}
		\label{fig:nat_interfaces}
	\end{figure}
	
	\subsubsection{Port Forwarding para Servicios Internos}
	\label{sec:lab8:portforwarding}
	Se configuró port forwarding para permitir que clientes externos accedieran
	a los servicios publicados por la organización en la VLAN de servicios. Para ello se ejecutan los comandos de la imagen \ref{fig:port_forwarding}.
	
	\begin{figure}[H]
		\centering
		\includegraphics[width=\textwidth]{\pathNat port forwarding.png}
		\caption{Reglas de port forwarding para publicar servicios internos.}
		\label{fig:port_forwarding}
	\end{figure}
	
	\section{Pruebas y Verificación}
	\label{sec:pruebas}
	
	A continuación de muestran distintas pruebas para comprobar el funcionamiento de la práctica:
	
	\subsection{Pruebas realizadas desde la VLAN 18}
	
	% --- Pruebas de conectividad entre VLANs ---
	\begin{figure}[H]
		\centering
		\includegraphics[width=\textwidth]{\pathTests PRUEBAS_PINGS_VLAN18.png}
		\caption{Pruebas de ping realizadas desde la VLAN 18.}
		\label{fig:pruebas_pings_vlan18}
	\end{figure}
	

	\subsection{Pruebas realizadas desde la VLAN 16}
	\begin{figure}[H]
		\centering
		\includegraphics[width=\textwidth]{\pathTests zbfw/vlan16tovlan16ok.png}
		\caption{Tráfico permitido dentro de la misma VLAN (VLAN16).}
		\label{fig:zbfw_vlan16_vlan16_ok}
	\end{figure}
	
	\begin{figure}[H]
		\centering
		\includegraphics[width=\textwidth]{\pathTests zbfw/vlan16_to_core.png}
		\caption{Prueba de conectividad desde VLAN16 hacia el core.}
		\label{fig:zbfw_vlan16_core}
	\end{figure}
	
	\begin{figure}[H]
		\centering
		\includegraphics[width=\textwidth]{\pathTests zbfw/vlan16_to_adm_and_serv.png}
		\caption{Prueba de conectividad desde VLAN16 hacia VLAN de administración y servicios.}
		\label{fig:zbfw_vlan16_adm_serv}
	\end{figure}

	\begin{figure}[H]
		\centering
		\includegraphics[width=\textwidth]{\pathTests zbfw/vlan16tovlan17and18.png}
		\caption{Comprobación de restricciones entre VLAN16, VLAN17 y VLAN18.}
		\label{fig:zbfw_vlan16_vlan17_18}
	\end{figure}
	
	\begin{figure}[H]
		\centering
		\includegraphics[width=\textwidth]{\pathTests zbfw/ping_vlan18_to_gateway.png}
		\caption{Ping desde VLAN18 al gateway, filtrado por el ZBFW.}
		\label{fig:zbfw_vlan18_gateway}
	\end{figure}
	
	
	\subsection{Comprobación de Reglas de Filtrado}
	\label{sec:pruebas:filtrado}
	
	\begin{itemize}
		\item \texttt{show access-lists}
		\item \texttt{show ip nat translations}
		\item \texttt{show zone-pair security}
		\item \texttt{show policy-map type inspect}
	\end{itemize}
	

	\section{Running-Configs}
	\label{sec:runningconfigs}
	
	\subsection{Configuración del Firewall (FW)}
	\label{sec:runningconfigs:fw}
	
	\begin{lstlisting}[language=bash, caption=Configuración del Firewall, label=lst:fw_config]
		% Configuración completa del firewall con ACLs, CBAC/ZBFW y NAT
	\end{lstlisting}
	
	\subsection{Configuración del Router CPE}
	\label{sec:runningconfigs:cpe}
	
	\begin{lstlisting}[language=bash, caption=Configuración del Router CPE, label=lst:cpe_config]
		% Configuración NAT y reglas de filtrado en CPE
	\end{lstlisting}
	
	\subsection{Configuración del Switch DL-SW1}
	\label{sec:runningconfigs:dlsw1}
	
	\begin{lstlisting}[language=bash, caption=Configuración del Switch DL-SW1, label=lst:dl_sw1_config]
		% Configuración de filtrado entre VLANs
	\end{lstlisting}
	
	\subsection{Configuración del Router ISP}
	\label{sec:runningconfigs:isp}
	
	\begin{lstlisting}[language=bash, caption=Configuración del Router ISP, label=lst:isp_config]
		% Configuración de enrutamiento y posibles ACLs
	\end{lstlisting}
	
\end{document}