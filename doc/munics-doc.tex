\documentclass{ltxdockit}
\usepackage{btxdockit}
\usepackage[utf8]{inputenc}
\usepackage[spanish]{babel}
\usepackage[strict]{csquotes}
\usepackage{tabularx}
\usepackage{longtable}
\usepackage{booktabs}
\usepackage{shortvrb}
\usepackage{pifont}
\usepackage{graphicx}
\usepackage{hyperref}
\usepackage{listings}
\usepackage{xcolor}
\usepackage{float}

% Definición de colores para código
\definecolor{codegreen}{rgb}{0,0.6,0}
\definecolor{codegray}{rgb}{0.5,0.5,0.5}
\definecolor{codepurple}{rgb}{0.58,0,0.82}
\definecolor{backcolour}{rgb}{0.95,0.95,0.92}

\lstdefinestyle{bashstyle}{
	backgroundcolor=\color{backcolour},   
	commentstyle=\color{codegreen},
	keywordstyle=\color{magenta},
	numberstyle=\tiny\color{codegray},
	stringstyle=\color{codepurple},
	basicstyle=\ttfamily\footnotesize,
	breakatwhitespace=false,         
	breaklines=true,                 
	captionpos=b,                    
	keepspaces=true,                 
	numbers=left,                    
	numbersep=5pt,                  
	showspaces=false,                
	showstringspaces=false,
	showtabs=false,                  
	tabsize=2,
	frame=single,
	language=bash
}

\lstset{style=bashstyle}

\newcommand*{\munics}{\emph{MUniCS Master's Thesis}\xspace}

\titlepage{%
	title={SEGURIDAD DE REDES},
	subtitle={Laboratorio 8 - Seguridad Perimetral},
	url={},
	author={Estela Pillo González \\ Orlando J. Garcés Casal \\ Simón Noya Dominguez \\ Alvaro Cainzos Urtiaga \\ Brais Gómez Espiñeira},
	email={estela.pgonzalez@udc.es \\ o.garces@udc.es \\ simon.noyad@udc.es \\ alvaro.cainzos@udc.es \\ brais.gomez2@udc.es},
	revision={1.0},
	date={\today}}

\hypersetup{%
	pdftitle={Documentación de Laboratorio 8 - Seguridad Perimetral},
	pdfsubject={Laboratorio 8 - Seguridad Perimetral},
	pdfauthor={Nombre del Estudiante},
	pdfkeywords={redes, seguridad, firewall, ACL, NAT, perimetral}}

\begin{document}
	
	\printtitlepage
	\tableofcontents
	
	\section{Introducción}
	\label{sec:intro}
	
	\subsection{Sobre esta Documentación}
	\label{sec:intro:about}
	
	Esta documentación recoge la implementación y resultados del Laboratorio 8 de Seguridad de Redes, centrado en la configuración de seguridad perimetral mediante firewalls, listas de control de acceso (ACL) y traducción de direcciones de red (NAT).
	
	\subsection{Topología y Objetivos}
	\label{sec:intro:topologia}
	
	Breve descripción de la topología física y lógica del laboratorio. Incluir aquí un diagrama general de la red con las diferentes zonas de seguridad (VLANs, Internet, etc.).
	
	\textbf{Objetivos del laboratorio:}
	\begin{itemize}
		\item Implementar políticas de seguridad perimetral mediante filtrado de paquetes.
		\item Configurar diferentes tipos de ACL (estándar, extendidas, basadas en zonas).
		\item Implementar CBAC (Context-Based Access Control) y ZBFW (Zone-Based Firewall).
		\item Configurar NAT (PAT dinámico y port forwarding).
		\item Comprender la interacción entre NAT y las reglas de filtrado.
		\item Garantizar el funcionamiento de protocolos de gestión y control de red.
	\end{itemize}
	
	\section{Laboratorio 8: Seguridad Perimetral}
	\label{sec:lab8}
	
	\subsection{Parte 1: Despliegue de Servicios}
	\label{sec:lab8:parte1}
	
	\subsubsection{Configuración de Servidores}
	\label{sec:lab8:servidores}
	
	Descripción general del despliegue de servicios en la VLAN de servicios:
	\begin{itemize}
		\item Servidor HTTP: dirección IP y puerto.
		\item Servidor HTTPS: dirección IP y puerto.
		\item Servidor DNS: dirección IP y puerto.
	\end{itemize}
	
	\textbf{Nota:} Indicar si se utilizan máquinas virtuales, contenedores, o software específico. Incluir capturas de configuración relevantes.
	
	\subsection{Parte 2: Configuración de Filtros en los Firewall}
	\label{sec:lab8:parte2}
	
	\subsubsection{Filtrado Estático de Paquetes}
	\label{sec:lab8:filtrado-estatico}
	
	Descripción de la ACL estándar implementada para bloquear direcciones IP no válidas o peligrosas:
	\begin{itemize}
		\item Direcciones privadas.
		\item Direcciones de enlace local.
		\item Direcciones multicast y broadcast.
		\item Loopback (127.0.0.0/8).
		\item Otras direcciones consideradas peligrosas (justificar).
	\end{itemize}
	
	Incluir aquí la configuración de la ACL estándar en el firewall.
	
	\subsubsection{Configuración de CBAC en FW}
	\label{sec:lab8:cbac}
	
	Implementación de la política de seguridad definida en la Tabla 1 mediante CBAC. Consideraciones:
	\begin{itemize}
		\item Tráfico bidireccional.
		\item Protocolos de enrutamiento (OSPF, etc.).
		\item Plano de gestión (SSH desde VLAN de administración).
		\item Acceso desde equipos de salto.
	\end{itemize}
	
	Incluir configuración de las ACLs extendidas y comandos CBAC aplicados.
	
	\subsubsection{Configuración de Zone-Based Firewall en FW}
	\label{sec:lab8:zbfw}
	
	Implementación alternativa de la política de seguridad mediante ZBFW:
	\begin{itemize}
		\item Definición de zonas (inside, outside, DMZ, etc.).
		\item Políticas entre zonas.
		\item Comparación CBAC vs ZBFW (ventajas/desventajas en esta topología).
	\end{itemize}
	
	\subsubsection{Filtrado en DL-SW1}
	\label{sec:lab8:filtrado-switch}
	
	Configuración de control de tráfico entre VLANs en el switch de distribución:
	\begin{itemize}
		\item Opciones de filtrado disponibles en el switch.
		\item Justificación de la opción elegida.
		\item Configuración aplicada (ACLs en capa 3, VACLs, etc.).
	\end{itemize}
	
	\subsection{Parte 3: Configuración de NAT}
	\label{sec:lab8:parte3}
	
	\subsubsection{PAT Dinámico para Tráfico Saliente}
	\label{sec:lab8:pat}
	
	Configuración de NAT overload (PAT) para las VLANs de usuario:
	\begin{itemize}
		\item VLAN 16: IP pública asignada.
		\item VLAN 17: IP pública asignada.
		\item VLAN 18: IP pública asignada.
	\end{itemize}
	
	Incluir comandos de configuración NAT en el router CPE.
	
	\subsubsection{Port Forwarding para Servicios Internos}
	\label{sec:lab8:portforwarding}
	
	Configuración de NAT estático para permitir acceso externo a servicios:
	\begin{itemize}
		\item Servicio HTTPS: mapeo de IP y puerto.
		\item Servicio HTTP: mapeo de IP y puerto.
	\end{itemize}
	
	\section{Pruebas y Verificación}
	\label{sec:pruebas}
	
	\subsection{Verificación de Conectividad}
	\label{sec:pruebas:conectividad}
	
	Descripción de pruebas realizadas para verificar:
	\begin{itemize}
		\item Acceso entre VLANs según política.
		\item Acceso a servicios desde Internet.
		\item Funcionamiento de NAT.
		\item Tráfico de gestión (SSH).
	\end{itemize}
	
	\subsection{Comprobación de Reglas de Filtrado}
	\label{sec:pruebas:filtrado}
	
	Uso de comandos como:
	\begin{itemize}
		\item \texttt{show access-lists}
		\item \texttt{show ip nat translations}
		\item \texttt{show zone-pair security}
		\item \texttt{show policy-map type inspect}
	\end{itemize}
	
	\section{Conclusiones}
	\label{sec:conclusiones}
	
	\subsection{Resultados Obtenidos}
	\label{sec:conclusiones:resultados}
	
	Resumen de los objetivos cumplidos y dificultades encontradas.
	
	\subsection{Comparativa CBAC vs ZBFW}
	\label{sec:conclusiones:comparativa}
	
	Análisis comparativo de ambas tecnologías en el contexto del laboratorio.
	
	\subsection{Reflexiones sobre Seguridad Perimetral}
	\label{sec:conclusiones:reflexiones}
	
	Consideraciones sobre la implementación de políticas de seguridad en entornos reales.
	
	\section{Running-Configs}
	\label{sec:runningconfigs}
	
	\subsection{Configuración del Firewall (FW)}
	\label{sec:runningconfigs:fw}
	
	\begin{lstlisting}[language=bash, caption=Configuración del Firewall, label=lst:fw_config]
		% Configuración completa del firewall con ACLs, CBAC/ZBFW y NAT
	\end{lstlisting}
	
	\subsection{Configuración del Router CPE}
	\label{sec:runningconfigs:cpe}
	
	\begin{lstlisting}[language=bash, caption=Configuración del Router CPE, label=lst:cpe_config]
		% Configuración NAT y reglas de filtrado en CPE
	\end{lstlisting}
	
	\subsection{Configuración del Switch DL-SW1}
	\label{sec:runningconfigs:dlsw1}
	
	\begin{lstlisting}[language=bash, caption=Configuración del Switch DL-SW1, label=lst:dl_sw1_config]
		% Configuración de filtrado entre VLANs
	\end{lstlisting}
	
	\subsection{Configuración del Router ISP}
	\label{sec:runningconfigs:isp}
	
	\begin{lstlisting}[language=bash, caption=Configuración del Router ISP, label=lst:isp_config]
		% Configuración de enrutamiento y posibles ACLs
	\end{lstlisting}
	
\end{document}