% !TEX root = my-thesis.tex


% **************************************************
% Files' Character Encoding
% **************************************************
%% Not necessary with luaLaTeX
% 
% \PassOptionsToPackage{utf8}{inputenc}
% \usepackage{inputenc}


% **************************************************
% Information and Commands for Reuse
% **************************************************
\newcommand{\thesisTitle}{Titulo}
\newcommand{\thesisName}{Subtitulo}
\newcommand{\thesisSubject}{Redes Seguras}
\newcommand{\thesisDate}{Curso 2022/2023}

\newcommand{\thesisFirstSupervisor}{}
\newcommand{\thesisSecondSupervisor}{}

\newcommand{\thesisUniversityStudies}{\protect{Máster Inter-Universitario en Ciberseguridade}}
\newcommand{\thesisUniversity}{Universidade da Coruña}     % Replace with your university
\newcommand{\thesisUniversitySchool}{Facultade de Informática} % Replace with your school
\newcommand{\thesisUniversityCity}{A Coruña}  % Replace with your city
\newcommand{\thesisUniversityStreetAddress}{Campus de Elviña s/n}
\newcommand{\thesisUniversityPostalCode}{15071}

% Poner los nombres de los alumnos en la portada, en el archivo content/titlepages.tex %

% **************************************************
% Debug LaTeX Information
% **************************************************
%\listfiles


% **************************************************
% Load and Configure Packages
% **************************************************
%\usepackage[spanish]{babel} % babel system, adjust the language of the content
\RequirePackage{polyglossia}
\setdefaultlanguage{spanish}
\setotherlanguage{english}

\PassOptionsToPackage{% setup clean thesis style
    figuresep=colon,%
    hangfigurecaption=false,%
    hangsection=true,%
    hangsubsection=true,%
    sansserif=false,%
    configurelistings=true,%
    colorize=full,%
    colortheme=bluemagenta,%
    configurebiblatex=true,%
    bibsys=biber,%
    bibfile=bib-refs,%
    bibstyle=alphabetic,%
    bibsorting=nty,%
}{munics}
\usepackage{munics}

\hypersetup{% setup the hyperref-package options
    pdftitle={\thesisTitle},    %   - title (PDF meta)
    pdfsubject={\thesisSubject},%   - subject (PDF meta)
    %pdfauthor={\thesisName},    %   - author (PDF meta)
    plainpages=false,           %   -
    colorlinks=false,           %   - colorize links?
    pdfborder={0 0 0},          %   -
    breaklinks=true,            %   - allow line break inside links
    bookmarksnumbered=true,     %
    bookmarksopen=true          %
}

% **************************************************
% Other Packages
% **************************************************
\usepackage{scrhack}
\usepackage{float}